\begin{figure}
\fbox{\textbf{Kinding:} $\Uty :: \kappa$}

\[
\begin{array}{lcl}
\texttt{data}~n~\overline{(a_i :: \kappa_i)} ~[\textsf{where}~ \I{dinfo}] &::& \overline{\kappa_i} \rightarrow \star
\\
\texttt{newtype}~n~\overline{(a_i :: \kappa_i)} = \tau &::& \overline{\kappa_i} \rightarrow \star
\\
\texttt{type}~n~\overline{(a_i :: \kappa_i)} :: \kappa = \tau &::& \overline{\kappa_i} \rightarrow \kappa
\\
\texttt{class}~n~\overline{(a_i :: \kappa_i)} ~[\textsf{where}~ \I{clinfo}] &::& \overline{\kappa_i} \rightarrow \textsf{Constraint}
\\
\texttt{type family}~n~\overline{(a_i :: \kappa_i)} :: \kappa ~[\texttt{where ..} \,|\, \I{tfinfo}] &::& \overline{\kappa_i} \rightarrow \kappa
\\
\texttt{type family}~n~\overline{(a_i :: \kappa_i)} &::& \overline{\kappa_i} \rightarrow \kappa
\\
n :: \tau &::& \star
\end{array}
\]
\caption{Defined entity kinding, where $\overline{\kappa_i} \rightarrow \kappa$ is interpreted as $\kappa_1 \rightarrow \kappa_2 \rightarrow \cdots \rightarrow \kappa$.  Note that no context is needed as the kind is fully encoded in declaration.}
\end{figure}
