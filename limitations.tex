\chapter{Limitations and future directions}
\label{sec:limitations}

\section{Metatheory}
\label{sec:metatheory}

We have not rigorously done proofs on \Backpack{}'s metatheory; as such,
we can only conjecture that in the fragment of Haskell without type
classes and type families, successful separate typechecking of
components implies successful linking as well.  Conventionally,
soundness proofs for module systems are done by elaboration to an
appropriate internal language; however, it is difficult to say what an
internal language which supports all of GHC Haskell's features would
look like, and in any case, a proof about this elaboration would say
little about the actual typechecking algorithm that GHC
implements.  \OldBackpack{}'s proof
of soundness gives evidence that \Backpack{} is sound, but we leave
an actual formalization to future work.

Note that we stated that \Backpack{} is sound \emph{without} type classes
and type families.  The soundness result we desire is not true for full
Haskell, and in fact, it cannot be made to be true.
Ordinary proofs of soundness rely on weakening: the ability of a signature
to hide information about its implementation.
Open type families~\cite{schrijvers+:typefamilies}
introduce axioms to Haskell which cannot be hidden by signatures:

\begin{lstlisting}
    signature A where
        f :: F Bool

    module B where
        import A
        type instance F Bool = Int -> Bool
        x = f 2
\end{lstlisting}
%
Now, should the following module be a permissible implementation of the
signature?

\begin{lstlisting}
    module A where
        type instance F Bool = Int
        f :: F Bool
        f = 42
\end{lstlisting}
%
Comparing only the signature and the module, it might seem that, yes,
the module is a permissible implementation, as it provides all of the
declarations required in the signature.  But if we do allow it,
then \verb|B| is clearly type unsafe, attempting to use an integer
as a function.

Is this problem fixable?  One ``fix'' is to require signatures to
specify every type family instance that an implementation could provide:
but it should be clear that this makes it essentially impossible to
write instances within a parametrized package.  Another fix is to change the rules for
permissible type family instances, so that instances defined in separate
modules are guaranteed not to conflict, no matter what. (\Red{give an
example?})  But this would constitute a fundamental change to how open
type families work in Haskell and rule out existing Haskell code.

The upshot is that, without changing how open type families work, we
must allow for the possibility that linking can fail, even if the
components typechecked separately.

There is a silver lining: the soundness of
\emph{compiled} \Backpack{} code reduces to the soundness of GHC
Haskell.  This is because we retypecheck and compile every instantiation
of a component, and thus the compilation process reduces to ordinary
Haskell compilation.  The retypechecking step during compilation is
where things like incompatible open type family axioms are found.
In general, if one avoids defining instance declarations for non-locally
defined types inside parametrized components, the problems with type
classes and open type families should not arise.  However, one should
still ask tough questions about the soundness of recursive modules;
and indeed, there is at least one known bug where \verb|hs-boot|
files can cause unsoundness.\footnote{\url{https://ghc.haskell.org/trac/ghc/ticket/9562}}

\section{Mutually recursive packages}

We have not described how to handle mutual recursion in \Backpack{},
although we believe that \Backpack{} can be extended to accommodate it,
based off of GHC's existing support for \verb|hs-boot| files.
Indeed, we showed in Section~\ref{sec:recursive-uids} that recursive
\uid{}s can be handled in the same way as they were done in \OldBackpack{}.

One challenge that we face is how to handle the ``double vision
problem'' in the presence of type synonyms.  \OldBackpack{} solved the
double vision problem for original names via  shaping pass; however,
languages like MixML which permit transparent type synonyms also have
required a type pre-computation pass to to solve double-vision at the
type level.

Interestingly, GHC's type checking algorithm is already designed in
such a way that it can solve the double vision problem in the following
way: an abstract type is opaque except within the \emph{implementing
module} (and any module which imports this module) at which point one
sees the value of the type synonym.

\begin{lstlisting}[language=Haskell]
    -- A.hs-boot
    module A where
        data T
        f :: T -> T
    -- B.hs
    module B(f) where
        import {-# SOURCE #-} A
    -- A.hs
    module A(T, f) where
        import qualified B
        type T = Int
        f :: Int -> Int
        f x = B.f x
\end{lstlisting}
%
When typechecking \verb|A.hs|, GHC resolves the type \verb|T| within
\verb|B.f| to be the type synonym \verb|type T = Int|, \emph{in the very
same module it is typechecking}.  In a sense, GHC has ``solved'' the
double vision problem, though it's worth noting that systems like RMC
and MixML address a stronger version of the double vision problem which
is inapplicable to Haskell due to the lack of sealing (see
Appendix~\ref{sec:double-vision}).  (In reality, GHC rejects \verb|A.hs|
because it does not allow abstract data to be implemented with type
synonyms, but in principle this restriction could be relaxed.)

\section{The signature language}

It is reasonably well understood how to reuse \emph{code}: functions and
values that need to be used multiple times are bundled into modules, and then
used by their downstream clients.  Less clear is how one goes about
reusing \emph{signatures}.

In \Backpack{}, we offer some limited facilities for signature reuse.
The most basic, essential form of signature reuse in a package language
based on mix-ins is \emph{signature merging}; this functionality is
essential for composing libraries with requirements, since without it,
we would have to manually rewrite the signature every time we put two
such libraries together.  In degenerate form, this allows us to place a
signature in a ``signature package'', to be reused by other packages
which depend on this package.

However, we have noticed that there are other operations on signatures
which end users might find useful:

\begin{enumerate}

    \item Given an implementation of a library, we might like to \emph{infer}
    the signature representing the API provided by this library,
    without having to copy paste the signatures of the library into a
    signature package of their own.

    \item A signature package may provide more functions than a client
    needs.  In this case, the client would prefer to \emph{thin} the
    functions from the signature package, rather than have to copy paste
    the subset they need into a signature of their own (especially if
    some of the types are quite large!)

    \item There might be multiple implementations of a particular
    interface; a client might be interested in taking the
    \emph{intersection} of these signatures, using only functionality
    that is available from all implementations.

\end{enumerate}
%
Each of these operations poses design questions which I do not have
satisfactory answers to.  For example, what should the syntax for signature
thinning be?  We might turn to the module export mechanism in Haskell today
for some ideas.  For example, here is the syntax for reexporting all the
exports of \verb|B|, except \verb|f|:

\begin{lstlisting}
    module A (module B) where
        import B hiding (f)
\end{lstlisting}
%
But this cannot be straightforwardly be adapted to signatures, where the
``reexports'' of inherited signatures happens implicitly: there is no
import statement to reexport!  Additionally, if a type is thinned from a
signature, we must transitively thin out any declarations which refer to
that type, and so forth.  Should this thinning occur automatically, or
should the user be obligated to explicitly thin them out?

We think there is an opportunity to design a more sophisticated language
of signatures, and we leave this as future work for \Backpack{}.
