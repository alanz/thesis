\documentclass{article}

\usepackage{mdframed}

%!TEX root = paper.tex
\usepackage[T1]{fontenc} % for tt curly braces
\usepackage{etoolbox} % for various programming stuff
\usepackage{amsmath}
\usepackage{listings}
\usepackage{array}
\usepackage{relsize}
\usepackage{amsthm}
\usepackage{amssymb}
\usepackage{color}
\usepackage[dvipsnames]{xcolor}
\usepackage{graphicx}
\usepackage{booktabs}
\usepackage{fancyvrb}
\usepackage{float}
\usepackage{stringstrings}
\usepackage{stmaryrd}
\usepackage{verbatimbox}
\usepackage{alltt}
\usepackage{multicol}
\usepackage{fixltx2e}
\usepackage[nooneline,bf]{subfigure}
% \usepackage[labelformat=simple]{subcaption}
% \renewcommand\thesubfigure{(\alph{subfigure})}
\usepackage{mdframed}
\usepackage{xcolor}
\usepackage{url}
\usepackage{xspace}
%\renewcommand*{\ttdefault}{txtt}

\lstset{basicstyle=\ttfamily,breaklines=true,columns=fullflexible}

\newmdenv[
    hidealllines=true,
    backgroundcolor=black!20,
    skipbelow=\baselineskip,
    skipabove=\baselineskip
]{greybox}

\newcommand{\ie}{\emph{i.e.},\xspace}
\newcommand{\eg}{\emph{e.g.},~}
\newcommand{\etal}{\emph{et al.}}

\theoremstyle{definition}
\newtheorem{axiom}{Axiom}
\newtheorem{theorem}{Theorem}
\newtheorem{definition}{Definition}
\newtheorem{lemma}{Lemma}

\newcommand{\I}[1]{\ensuremath{\mathit{#1}}}
\newcommand{\cL}{{\calL}}
\newcommand{\Red}[1]{{\color{red} #1}}
\newcommand{\Blue}[1]{{\color{blue} #1}}
\newcommand{\Updated}[1]{\colorbox{red}{Updated! #1}}

% \newcommand{\Backpack}{\texttt{i}Backpack}%
\newcommand{\OldBackpack}{Backpack'14}%
\newcommand{\Backpack}{Backpack'16}%
\newcommand{\Ccomp}{Resolved component}%
\newcommand{\ccomp}{resolved component}%
\newcommand{\Unit}{Mixed component}%
\newcommand{\unit}{mixed component}%
\newcommand{\Iunit}{Instantiated component}%
\newcommand{\iunit}{instantiated component}%
\newcommand{\cid}{component identifier}%
\newcommand{\Cid}{Component identifier}%
\newcommand{\uid}{unit identifier}%
\newcommand{\Uid}{Unit identifier}%
\newcommand{\ir}{linked intermediate representation}
\newcommand{\Ir}{Linked intermediate representation}

% macros for unit syntax and semantic objects
% ------------------------------
% metavars
\newcommand{\Up}{p}
\newcommand{\UP}{P}
\newcommand{\Un}{n}
\newcommand{\UN}{N}
\newcommand{\UNs}{\mathit{Ns}}
\newcommand{\USn}{\mathit{Sn}}
\newcommand{\Ureqs}{\mathit{mreqs}}
\newcommand{\Uunit}{\mathit{mcomp}}
\newcommand{\mprog}{\mathit{mprog}}
\newcommand{\mcomp}{\mathit{mcomp}}
%\newcommand{\Uudecls}{\mathit{ldecls}}
\newcommand{\Uudecl}{\mathit{mdecl}}
\newcommand{\mdecl}{\mathit{mdecl}}
\newcommand{\mdecls}{\mathit{mdecls}}
\newcommand{\marg}{\mathit{marg}}
\newcommand{\Uhsexps}{\mathit{hsexps}}
\newcommand{\Uhsbody}{\mathit{hsbody}}
\newcommand{\Uhssig}{\mathit{hssig}}
\newcommand{\Uunitty}{\Xi}
\newcommand{\Umty}{T}
\newcommand{\Urole}{R}
\newcommand{\Uty}{\mathit{decl}}
\newcommand{\Utys}{\mathit{decls}}
\newcommand{\Udfun}{\texttt{\$} \I{fn}}
\newcommand{\Uinst}{\mathit{inst}}
\newcommand{\Uinsts}{\mathit{insts}}
\newcommand{\Uimp}{\mathit{imp}}
\newcommand{\Uimps}{\mathit{imps}}
\newcommand{\Ufinst}{\mathit{finst}}
\newcommand{\Ufinsts}{\mathit{finsts}}
\newcommand{\Umatch}{\mathit{sub}}
% syntax
\newcommand{\UsynUnitH}[2]{\ensuremath{\texttt{component}~#1~#2}}
\newcommand{\UsynUnit}[3]{\ensuremath{\UsynUnitH{#1}{#2}~\{#3\}}}
\newcommand{\UsynMod}[2]{\ensuremath{\texttt{module}~\modnameif{#1}~\{~#2~\}}}
\newcommand{\UsynSig}[2]{\ensuremath{\texttt{signature}~\modnameif{#1}~\{~#2~\}}}
\newcommand{\UsynSigWith}[3]{\ensuremath{\texttt{signature}~\modnameif{#1}~\{~#2~\}~\textsf{with}~#3}}
\newcommand{\UsynDep}[2]{\ensuremath{\texttt{dependency}~#1~(#2)}}
\newcommand{\UsynDepEmpty}[1]{\ensuremath{\texttt{dependency}~#1}}
% semantic objects
\newcommand{\UobjIface}{\texttt{iface}}
\newcommand{\UobjTau}[2]{\ensuremath{\UobjIface\:(#1)~\{#2\}}}
\newcommand{\UobjUnitTy}[4]{\ensuremath{\forall~#1.\, \forall~#2 .\, #3 \rightarrow #4}}
\newcommand{\UobjUnitTyN}[3]{\ensuremath{\forall~#1 .\, #2 \rightarrow #3}}
% ------------------------------



\newcommand{\doubleplus}{\ensuremath{\mathbin{+\mkern-8mu+}}}
\newcommand{\scomp}[2]{\ensuremath{#2 \circ #1}}

\newcommand{\modname}[1]{\ensuremath{\texttt{#1}}}%
\newcommand{\modnameif}[1]{%
  % make the argument a modname if it is not a macro (invocation)
  % and if it is capitalized
  \ifdefmacro{#1}%
    {#1}%
    {%
      \testcapitalized{#1}
      \ifcapitalized 
        \modname{#1}
      \else
        #1
      \fi%
    }%
  % % make the argument a modname if it is not a (primed) "m" or "n" metavar
  % \ifboolexpr{ test {\ifstrequal{#1}{m}} or %
  %              test {\ifstrequal{#1}{m'}} or %
  %              test {\ifstrequal{#1}{m''}} or %
  %              test {\ifstrequal{#1}{n}} or %
  %              test {\ifstrequal{#1}{n'}} or %
  %              test {\ifstrequal{#1}{n''}} }%
  %   {#1}
  %   {\modname{#1}}
}


\newcommand{\hole}[1]{\ensuremath{#1}}%
% \newcommand{\holevar}[1]{\ensuremath{\colorbox{Apricot}{\ensuremath{#1}}}}%
% \newcommand{\holevar}[1]{\ensuremath{\colorbox{Apricot}{\ensuremath{\modnameif{#1}}}}}
\newcommand{\holevar}[1]{\ensuremath{\langle\modnameif{#1}\rangle}}
\newcommand{\nholevar}[1]{\ensuremath{\{\modnameif{#1}\}}}
\newcommand{\hv}[1]{\holevar{#1}}
\newcommand{\nhv}[1]{\nholevar{#1}}

\newcommand{\cidl}[1]{\ensuremath{\mbox{\texttt{#1}}}}%
\newcommand{\cidlif}[1]{
  % make the argument a component ID if it is not a (primed) "p" or "q" metavar
  % NOTE: sometimes "p" is used as a literal! in such cases, use \cidl instead
  \ifdefmacro{#1}%
    {#1}%
    {%
      \ifboolexpr{ test {\ifstrequal{#1}{p}} or %
                   test {\ifstrequal{#1}{p'}} or %
                   test {\ifstrequal{#1}{p''}} or %
                   test {\ifstrequal{#1}{q}} or %
                   test {\ifstrequal{#1}{q'}} or %
                   test {\ifstrequal{#1}{q''}} }%
        {#1}
        {\cidl{#1}}%
    }
}
\newcommand{\icid}[2]{\cidlif{#1}[#2]}
\newcommand{\uidl}[2]{%
  \ensuremath{
    \cidl{#1}%
    [%
      % if subst is non-empty, allow line break after open paren
      \ifstrempty{#2}{}{\allowbreak}%
      #2
    ]%
  }}

% \newcommand{\Mod}[2]{\ensuremath{#1\!:\!#2}}%
\newcommand{\modcolon}{\hspace{.1ex}{:}\hspace{.15ex}}
\newcommand{\Mod}[2]{\ensuremath{#1\modcolon\modnameif{#2}}}%
\newcommand{\MOD}[3]{\ensuremath{\icid{\cidl{#1}}{#2}\modcolon\modnameif{#3}}}%
\newcommand{\subst}[2]{\ensuremath{\modnameif{#1} \mathop{=}\allowbreak #2}}
\newcommand{\substMod}[3]{\subst{#1}{\Mod{#2}{#3}}}
\newcommand{\substMOD}[4]{\subst{#1}{\MOD{#2}{#3}{#4}}}
\newcommand{\substHole}[1]{\subst{#1}{\holevar{#1}}}
\newcommand{\substw}[2]{#1\llbracket#2\rrbracket}
\newcommand{\substww}[3]{#1\llbracket#2\rrbracket\llbracket#3\rrbracket}
\newcommand{\prov}[2]{\ensuremath{\modnameif{#1} \,\mapsto\,\allowbreak #2}}
\newcommand{\provMod}[3]{\prov{#1}{\Mod{#2}{#3}}}
\newcommand{\provMOD}[4]{\prov{#1}{\MOD{#2}{#3}{#4}}}
\newcommand{\provHole}[1]{\prov{#1}{\holevar{#1}}}
\newcommand{\rename}[2]{\ensuremath{\modnameif{#1} \mapsto \modnameif{#2}}}


% now render unit IDs like
%   \uidl{mylib-3.0-aaa}{
%     \subst{m}{M},                                         % m and M are metavars
%     \substMod{Database}{\uidl{mysql-1.0-ccc}{}}{MySQL}    % Database, mysql..., MySQL are rendered
%    }


\newcommand{\haspr}{\:\triangleright\:}
\newcommand{\shapeis}{\triangleright}
%\newcommand{\lctx}{\mathcal{L}}
\newcommand{\lctx}{{\tilde{\Xi}}}
%\newcommand{\provs}{\mathcal{L}_\textsf{prov}}
%\newcommand{\provs}{{\tilde{\Sigma}}_\textsf{P}}
\newcommand{\provs}{{\tilde{\Sigma}}}
\newcommand{\impctx}{\mathcal{L}}
%\newcommand{\reqs}{{\tilde{\Sigma}}_\textsf{R}}
\newcommand{\reqs}{\Theta}
\newcommand{\pre}[1]{\tilde{#1}}
\newcommand{\preprovs}{\mathbb{P}}
\newcommand{\prereqs}{\mathbb{R}}
\newcommand{\preshape}{\mathbb{L}}
\newcommand{\Reqs}{\mathcal{R}}
\newcommand{\preP}{\preprovs}
\newcommand{\preR}{\prereqs}
\newcommand{\preL}{\preshape}
\newcommand{\shctx}{\tilde{\Gamma}}
%\newcommand{\lctxpair}[2]{\langle #1 \,;\, #2 \rangle}
\newcommand{\lctxpair}[2]{\forall #2.\, #1}
%\newcommand{\lctxpair}[2]{\lctxpairx{#1}{#2}}
\newcommand{\lctxpairx}[2]{\forall #2.\, \{ #1 \}}
%\newcommand{\lctxpairx}[2]{\begin{pmatrix}
%\textsf{requires:}& #2 \\
%\textsf{provides:}& #1 \\
%\end{pmatrix}}
\newcommand{\lctxpairex}[2]{\{#2\} &\rightarrow& \{#1\}}
\newcommand{\lift}[1]{\uparrow\!#1}
\newcommand{\shnull}{}
%\newcommand{\shnull}{\cdot}
%\newcommand{\shape}{\tilde{\Xi}}

% curly braces
% \newcommand{\lcb}{{\tt {\char 123}}}
% \newcommand{\rcb}{{\tt {\char 125}}}
\newcommand{\wrapcb}[1]{\texttt{\textbraceleft} #1 \texttt{\textbraceright}}

% S stands for "source", but these are not the true source declarations.

\newcommand{\Scomp}{\I{rcomp}}
\newcommand{\Sincl}{\I{rincl}}
\newcommand{\Sdecl}{\I{rdecl}}
\newcommand{\Scomponent}[2]{\texttt{component}~ #1 ~ \wrapcb{#2}}
\newcommand{\Sinclude}[2]{\texttt{mixin:}~ #1 ~#2}
\newcommand{\Sincludespec}[3]{\texttt{mixin:}~ #1 ~[\texttt{(}#2\texttt{)}]~[\texttt{requires}~ \texttt{(}#3\texttt{)}]}
\newcommand{\Slparen}{\texttt{(}}
\newcommand{\Srparen}{\texttt{)}}
\newcommand{\Sexposed}[2]{\texttt{exposed-module:}~ \modnameif{#1} ~\{ #2 \}}
\newcommand{\Sother}[2]{\texttt{other-module:}~ \modnameif{#1} ~\{ #2 \}}
\newcommand{\Srequired}[2]{\texttt{signature:}~ \modnameif{#1} ~\{ #2 \}}
\newcommand{\Sreexported}[2]{\texttt{reexported-module:}~ \modnameif{#1} ~\texttt{as}~ \modnameif{#2}}
\newcommand{\Sas}[2]{\modnameif{#1} \;\texttt{as}\; \modnameif{#2}}
\newcommand{\Scom}{\texttt{,}\,}

% macros for helper functions
\newcommand{\Freqs}[2][\Delta]{\ensuremath{\mathsf{reqnames}_{#1}(#2)}}
\newcommand{\Fpreshape}[1]{\ensuremath{\mathsf{preshape}(#1)}}
\newcommand{\FpreshapeI}[2][\Delta]{\ensuremath{\mathsf{preshape}_{#1}(#2)}}
\newcommand{\Fprelink}[1]{\ensuremath{\mathsf{merge}(#1)}}
\newcommand{\FprelinkTwo}[2]{\ensuremath{\mathsf{merge}(#1; #2)}}
\newcommand{\Flink}[1]{\ensuremath{\mathsf{link}(#1)}}
\newcommand{\Fsource}[1]{\ensuremath{\mathsf{source}(#1)}}
\newcommand{\Flet}[1]{\ensuremath{\mathsf{bind}(#1)}}
\newcommand{\Fprerename}[3]{\ensuremath{\mathsf{rnthin}(#1; #2; #3)}}
\newcommand{\Frename}[3]{\ensuremath{\mathsf{rnthin}(#1; #2; #3)}}
\newcommand{\Fprovs}[1]{\ensuremath{\mathsf{provs}(#1)}}
\newcommand{\Fdom}[1]{\ensuremath{\mathsf{dom}(#1)}}
\newcommand{\Fexports}[1]{\ensuremath{\mathsf{exps}(#1)}}
\newcommand{\Ftrim}[2]{\ensuremath{\mathsf{trim}_{#1}(#2)}}
\newcommand{\Ffnv}[1]{\ensuremath{\mathsf{fnv}(#1)}}

% notational convenience
\newcommand{\overlinei}[1]{\overline{#1}^i}
\newcommand{\overlinej}[1]{\overline{#1}^j}
\newcommand{\overlinek}[1]{\overline{#1}^k}
\newcommand{\overlinel}[1]{\overline{#1}^l}



\renewcommand{\topfraction}{.85}
\renewcommand{\bottomfraction}{.7}
\renewcommand{\textfraction}{.1}
\renewcommand{\floatpagefraction}{.9}
\renewcommand{\dbltopfraction}{.9}
\renewcommand{\dblfloatpagefraction}{.9}
\setcounter{topnumber}{9}
\setcounter{bottomnumber}{9}
\setcounter{totalnumber}{20}
\setcounter{dbltopnumber}{9}


\newcommand{\Edward}[1]{\textcolor{Red}{\bf[EZY: #1]}}
\newcommand{\Scott}[1]{\textcolor{Plum}{\bf[SK: #1]}}
\newcommand{\Derek}[1]{\textcolor{Blue}{\bf[DD: #1]}}
\newcommand{\Simon}[1]{\textcolor{RedOrange}{\bf[SPJ: #1]}}
\newcommand{\simon}[1]{\Simon{#1}}

% macros for examples

\newcommand{\uidArraysA}{%
  \uidl{arrays-a}{\holevar{Prelude}}%
}
\newcommand{\uidArraysB}{%
  \uidl{arrays-b}{\holevar{Prelude}}%
}
\newcommand{\uidStructures}{%
  \uidl{structures}{\holevar{Prelude}, \holevar{Array}}%
}
\newcommand{\uidStructuresA}{%
  \uidl{structures}{\holevar{Prelude},
                    \Mod{\uidArraysA}{Array}}%
}
\newcommand{\uidStructuresB}{%
  \uidl{structures}{\holevar{Prelude},
                    \Mod{\uidArraysB}{Array}}%
}

\newcommand{\Tmatch}[3]{#1 <:_{#2} #3}
\newcommand{\Tsubg}[3]{\{#1\} \quad #2 \quad \{ #3 \}}
\newcommand{\Tsub}[5]{\{#1\} \quad \Tmatch{#2}{#3}{#4} \quad \{ #5 \}}


\newcommand{\highlight}[1]{%
  \colorbox{yellow!50}{$\displaystyle#1$}}

\newcommand{\graybox}[1]{%
  \colorbox{gray!30}{$\displaystyle#1$}}

\newcommand{\KP}{\mathsf{P}}
\newcommand{\KM}{\mathsf{M}}

\newcommand{\RP}{\mathbb{P}}
\newcommand{\RM}{\mathbb{M}}
\newcommand{\RN}{\mathbb{N}}

\newcommand{\RT}{T}
\newcommand{\RE}{E}

\newcommand{\row}{\textsf{row}}
\newcommand{\rowty}[1]{\llparenthesis{}\,#1\,\rrparenthesis{}}
\newcommand{\record}[1]{\textsf{record}~#1}


\newcommand{\elab}[1]{\ensuremath{\leadsto{} #1}}


\newcommand{\CK}{\mathcal{K}}
\newcommand{\CT}{\mathcal{T}}

\newcommand{\Ictx}{\Gamma_c}
\newcommand{\IGamma}{\Gamma_c}
\newcommand{\IDelta}{\Delta_c}
\newcommand{\Idash}{\vdash_c}
\newcommand{\Impctx}{\impctx_c}
\newcommand{\iprog}{\I{iprog}}
\newcommand{\invariant}{\prec}

\newcommand{\tDelta}{\tilde{\Delta}}
\newcommand{\Quant}{\Theta}
\newcommand{\induces}{\Rightarrow}
% \newcommand{\declst}[3]{\begin{Bmatrix}#1\\#2\\#3\end{Bmatrix}}
\newcommand{\declst}[3]{\begin{Bmatrix}#2\\#3\end{Bmatrix}}
\newcommand{\Ideclst}[2]{\begin{Bmatrix}#1\\#2\end{Bmatrix}}
%\newcommand{\hsubis}{\rightarrow_S}
%\newcommand{\nhsubis}{\rightarrow_{\USn}}


\newcommand{\JMatch}[6]%
  [\Gamma; \Theta; \theta; \Delta; \Up_0]%
  {\ensuremath{#1 \vdash #2 : \forall #3.\, \forall #4.\, #5 \induces #6}}

\newcommand{\JReqType}[4]%
  % [\Gamma; \Up_0; \Theta; \theta; \Delta]%
  [\Gamma; \Theta; \theta; \Delta; \Up_0]%
  {\ensuremath{#1 \vdash #2 \,@\, #3 : #4}}

\newcommand{\thetaOR}{\hat\theta}
\newcommand{\SigmaOR}{\hat\Sigma}

% Alter some LaTeX defaults for better treatment of figures:
    % See p.105 of "TeX Unbound" for suggested values.
    % See pp. 199-200 of Lamport's "LaTeX" book for details.
    %   General parameters, for ALL pages:
    \renewcommand{\topfraction}{0.9}	% max fraction of floats at top
    \renewcommand{\bottomfraction}{0.8}	% max fraction of floats at bottom
    %   Parameters for TEXT pages (not float pages):
    \setcounter{topnumber}{2}
    \setcounter{bottomnumber}{2}
    \setcounter{totalnumber}{4}     % 2 may work better
    \setcounter{dbltopnumber}{2}    % for 2-column pages
    \renewcommand{\dbltopfraction}{0.9}	% fit big float above 2-col. text
    \renewcommand{\textfraction}{0.07}	% allow minimal text w. figs
    %   Parameters for FLOAT pages (not text pages):
    \renewcommand{\floatpagefraction}{0.7}	% require fuller float pages
	% N.B.: floatpagefraction MUST be less than topfraction !!
    \renewcommand{\dblfloatpagefraction}{0.7}	% require fuller float pages

	% remember to use [htp] or [htpb] for placement
        %
\lstset{language=Haskell,keywords={%  
    package, module, signature, link, unit, where, data, import, instance, hiding, type, dependency%
    }%
}%
\lstdefinelanguage{Cabal}
{
  % list of keywords
  morekeywords={
    name,version,library,exposed-modules,signatures,build-depends,main-is,executable,mixins
  },
  alsoletter=-,
  sensitive=false, % keywords are not case-sensitive
  morecomment=[l]{--}, % l is for line comment
  morestring=[b]" % defines that strings are enclosed in double quotes
}


\newcommand{\DIGatoms}{%
    \begin{array}{rcl}
    m & & \mbox{Module name} \\
    p, q & & \mbox{\Cid} \\
    \end{array}
}

\newcommand{\DIGsource}{%
    \begin{array}{rcl}
    \Uhsbody & & \mbox{Module source} \\
    \Uhssig & & \mbox{Signature source} \\
    \end{array}
}

\newcommand{\DIGresolved}{%
    \begin{array}{rcl}
    \Scomp & ::= & \Scomponent{\Up}{\overline{\Sdecl}} \\[0.3em]
    \Sdecl & ::= & \Sinclude{\Up}{\I{rns}} \\
           & |   & \Sexposed{m}{\Uhsbody} \\
           & |   & \Sother{m}{\Uhsbody} \\
           & |   & \Sreexported{m}{m'} \\
           & |   & \Srequired{m}{\Uhssig} \\
    \I{rns} & ::= & [\Slparen\overline{rn}\Srparen]~[\texttt{requires}~ \Slparen\overline{rn'}\Srparen] \\
    \I{rn} & ::= & \Sas{m}{m'} \\
           & |   & m \\
    \end{array}
}

\newcommand{\DIGuid}{%
    \begin{array}{rcll}
      M   &::=& \Mod{\UP}{m} & \text{Module identifier} \\
          &|&   \holevar{m} & \text{Module hole} \\
      \UP &::=& \icid{\Up}{S} & \text{\Uid} \\
      S   &::=& \overline{\subst{m}{M}} & \text{Module substitution} \\
    \end{array}
}

\newcommand{\DIGshape}{%
    \begin{array}{lcll}
    \lctx & ::= & \lctxpairx{\provs}{\reqs} & \mbox{Component shape} \\
    %r_\textsf{R} & ::= & \overline{m \rightarrow m'} & \mbox{Requirement renaming (total)} \\
    %r_\textsf{P} & ::= & \overline{m \rightarrow m'} & \mbox{Provision renaming (partial)} \\
    \reqs  & ::= & \overline{\hv{m}} & \mbox{Required module variables} \\
    \provs & ::= & \overline{\prov{m}{M}}  & \mbox{Provided modules} \\
    \shctx & ::= & \overline{\Up \shapeis \lctx} & \mbox{Component shape context} \\
    \end{array}
}

\newcommand{\DIGmixed}{%
    \begin{array}{rcl}
      \Uunit &::=& \UsynUnit{\Up}{\reqs}{\overline{\Uudecl}} \\
      \Uudecl &::=& \UsynDep{\UP}{r} \\
              &|&   \UsynMod{m}{\Uhsbody} \\
              &|&   \UsynSig{m}{\Uhssig} \\
              %& &   \qquad \textsf{with}~\overline{P\!:\!m} & \quad\text{(Merges)} \\
              %&|&   \UsynLet{m}{M} & \text{Let binding} \\
      %\mprog &::=& \overline{\Uunit} & \text{Mixed program} \\
      r   &::=& \overline{m \mapsto m'} \\
    \end{array}
}

\newcommand{\DIGinterface}{
\begin{array}{rcll}
  %\Uunitty &::=& \forall \Theta.\, \forall \theta.\, \{\Sigma_R\} \rightarrow \{\Sigma_P\}
  %  & \text{Component type} \\
  %\theta &::=& \overline{\hv{m.n}} & \text{Name variable quantifiers} \\
  %\Sigma &::=& \overline{m : \Umty}
  %  & \text{Provided/required types} \\
  %\Gamma &::=& \overline{\Up : \Uunitty} & \text{Component typing context} \\
  \Gamma &::=& \overline{p : \Xi} & \text{Package environment} \\
  \Delta, \Xi &::=& \overline{m : T^s} & \text{Package type} \\
  &&&\\
  \Umty &::=& \UobjTau{\UNs}{\overline{\Uty}~\overline{\Uinst}~\overline{\Uimp}} & \text{Module type} \\
  s &:=& + ~|~ -  & \text{Polarities} \\
  % \tau_R &::=& \exists \overline{\hv{m.n}}.\, \tau  & \text{Signature type} \\
  \UNs &::=& \overline{\UN} & \text{Export specification} \\
  \Uty &::=& & \text{Defined entity spec} \\
       &  |& \texttt{data}~\graybox{n}~\overline{(a :: \kappa)} & \qquad\text{Abstract data declaration} \\
       &  |& \texttt{class}~\graybox{n}~\overline{(a :: \kappa)} & \qquad\text{Abstract type class} \\
       &  |& \texttt{type family}~\graybox{n}~\overline{(a :: \kappa)} :: \kappa~\texttt{where}~\texttt{..} & \qquad\text{Abstract closed type family} \\
       &  |& \texttt{data}~\graybox{n}~\overline{(a :: \kappa)}~\texttt{where}~\I{dinfo}& \qquad\text{Data declaration} \\
       &  |& \texttt{newtype}~\graybox{n}~\overline{(a :: \kappa)} = \I{ntinfo}& \qquad\text{Newtype declaration} \\
       &  |& \texttt{type}~\graybox{n}~\overline{(a :: \kappa)} :: \kappa = \tau & \qquad\text{Type synonym} \\
       &  |& \texttt{class}~\graybox{n}~\overline{(a :: \kappa)}~\texttt{where}~\I{clinfo} & \qquad\text{Type class} \\
       &  |& \texttt{type family}~\graybox{n}~\overline{(a :: \kappa)} :: \kappa~\texttt{where}~\I{tfinfo} & \qquad\text{Closed type family} \\
       &  |& \texttt{type family}~\graybox{n}~\overline{(a :: \kappa)} :: \kappa & \qquad\text{Open type family} \\
       % Patterns omitted
       % Data families omitted
       &  |& n :: \tau& \qquad\text{Term declaration} \\
  % Notes: Why are the axioms/names kept separately from the instance
  % lists, rather than embedded "implicitly" (e.g., like how coercion
  % for newtype is stored)?  Historically, the DFun was always kept
  % separate from the instance declaration.
  \Uinst &::=& \texttt{instance}~n :: \tau & \text{Class instance} \\
         &  |& \texttt{family instance}~n :: \I{fiinfo} & \text{Family instance} \\
  \Uimp  &::=& M & \text{Transitive module import} \\
  &&&\\
  \Un   && \multicolumn{2}{l}{\text{Haskell source-level entity name}} \\
  \UN &::=& M.\Un & \text{Original name} \\
      &|&   \nhv{m.n} & \text{Name hole} \\
  \USn &::=& \overline{\subst{m.n}{N}} & \text{Name substitution} \\
  &&&\\
  a &&& \text{Haskell type variable} \\
  \tau &::=& \cdots N \cdots & \text{Haskell type} \\
  \kappa &::=& \cdots N \cdots & \text{Haskell kind} \\
  \multicolumn{3}{l}{\I{dinfo}, \I{ntinfo}, \I{clinfo}, \I{tfinfo}, \I{fiinfo}} & \text{Haskell declaration metadata} \\
% R &::=& & \text{Haskell role} \\
%   & | & \mathsf{P} & \qquad\text{Phantom} \\
%   & | & \mathsf{R} & \qquad\text{Representational} \\
%   & | & \mathsf{N} & \qquad\text{Nominal} \\
\end{array}
}


\DeclareMathOperator{\dom}{dom}

\newcommand{\ctx}{\Gamma; \Delta; P_0}

\newenvironment{twocol}
  {
    \setlength{\abovedisplayskip}{0pt}
    \setlength{\belowdisplayskip}{0pt}
    \begin{tabular}{ m{0.47\textwidth} m{0.47\textwidth} }
  }
  { \end{tabular} }

\newcommand\defeq{\mathrel{\overset{\makebox[0pt]{\mbox{\normalfont\tiny def}}}{=}}}


\begin{document}

\special{papersize=8.5in,11in}
\setlength{\pdfpageheight}{\paperheight}
\setlength{\pdfpagewidth}{\paperwidth}

\title{F-ing Backpack}

\maketitle

\noindent

In this section, I present an F-ing style semantics for Backpack on an
extremely small subset of the Haskell language (no type classes, no
recursion, post renaming).  This semantics demonstrates that Backpack
can be interpreted as a module system of \emph{applicative functors},
and subsequently, its semantics can be defined via an elaboration to
System F$\omega$.

Our elaboration is on a tremendously simplified version of Haskell,
so simplified, indeed, that one might wonder, ``What is the point?''
Nor is the elaboration at all reflective of our actual implementation
of typechecking in GHC proper.

To this, I offer the following justifications:

\begin{itemize}

\item It serves as a sound declarative specification of how Backpack in
    GHC Haskell should behave, when only the features supported in our
    subset our used.  As an elaboration to a well-known target calculus,
    it does not require any complex metatheory and can be easily compared
    to other ML module systems.  The semantic objects of the elaboration
    also allow us to develop some metatheory about \emph{signature merging}.

\item Our language is very simple, and could be used as a tutorial introduction
    to F-ing style elaborations with applicative functors.
    Our formalization does not substantially differ from the one
    presented in F-ing modules, but Backpack does not have generative
    functors, allowing us to remove one of the devices used in the
    F-ing paper to support generative and applicative functors simultaneously.

\item GHC Haskell is an extremely complex language, which has never
    been formalized in its entirety, never mind formalizing it with
    Backpack.  A line must be drawn somewhere, and a language with
    abstract types, type synonyms and function declarations is a fine
    place to start.

\item Even if it were possible to formalize all of GHC Haskell, we would
    unable to provide a compositional elaboration of the resulting language,
    as certain popular features in Haskell (open type families, in particular)
    are inherently non-compositional.  Perhaps this indicates a failure
    of judgment in language design, but the assumption that instances
    are globally unique has been a convenient one, which is baked
    into much of the code that has been written in Haskell.

\end{itemize}

\section{Explicitly typed, renamed Backpack}

In this section, we describe an explicitly typed,
renamed version of Backpack which we call ``Explicit Backpack'':

\begin{itemize}
    \item It is \emph{explicitly typed}, in that all components are
    annotated with a complete, fully merged signature for each of
    its requirements (unlike \textsf{signature} declarations in
    regular Backpack, which simply add to the inferred requirements
    of a component), and
    \item It is \emph{renamed}, meaning that source-level names like
    \texttt{x} and \texttt{T} have been resolved to their actual
    definition sites.  In renamed code, there is no need for \texttt{import}
    statements, export lists
    or \textsf{dependency} declarations: instead, names refer directly
    to entities defined in the context.\footnote{Technically, we also
    need the capacity to assert that a dependency is well-typed, to
    handle instantiations from mixin linking which don't show up in
    the renamed code.  However, this is an easy extension to the
    module language, and for simplicity we don't include it for now.}
\end{itemize}

\noindent
Effectively, it is a very small module system which only supports
\emph{applicative functors} (no sealing, no hierarchical modules, no
\textsf{include} statement, no higher-order functors).

\section{Syntax}

\paragraph{Components}
The top level construct in Explicit Backpack is a \emph{component},
i.e., an applicative functor:

\begin{figure}[H]
\[
\begin{array}{rcll}
\I{rprog} &::=& \overline{p = \I{rcomp}} \\
\I{rcomp} &::=& \lambda [ \I{rsigs} ].\, \{ \I{rmods} \} & \mbox{Component} \\
\\
\I{rsigs} &::=& \overline{m : \{ \I{rsig} \}} & \mbox{Signature bindings} \\
\I{rsig}  &::=& \overline{\I{rsigd}} & \mbox{Signature} \\
\I{rsigd}  &::=& \mathbf{data}~n & \mbox{Signature declaration} \\
           & | & \mathbf{data}~n = n_c~\{ n_d :: T \} \\
           & | & \mathbf{type}~n = T \\
           & | & n :: \RT \\
\\
\I{rmods} &::=& \overline{m = \{ \I{rmod} \}} & \\
\I{rmod} &::=& \overline{\I{rmodd}} & \mbox{Module} \\
\I{rmodd} &::=& \mathbf{type}~n = T & \mbox{Module declaration} \\
          & | & \mathbf{data}~n = n_c~\{ n_d :: T \} \\
          & | & n = \RE \\
\end{array}
\]
\caption{Syntax for renamed components}
\end{figure}

\noindent
A component is a sequence of signatures (constituting the arguments
of the functor), and then a sequence of modules defined by the component
(the body of the functor.)  Signature and module bindings bring modules
into scope, which can be referred to in expressions $E$ and types $T$;
additionally, we may \emph{instantiate} a component (functor application)
in order to get modules from an external component.

\paragraph{What happened to the algebraic data types?}
For simplicity, in this formalization we consider \emph{restricted
algebraic data types}: those which could be equivalently have been
written as \textbf{newtype}s.  We don't consider \emph{full} algebraic
data types, as they would have to be encoded into $\mathrm{F}_\omega$
(or the calculus extended with explicit support for them, as is the case
for System Fc).  This is besides the point of this F-ing semantics, so
we don't attempt to handle it.  Similarly, we don't bother
defining pattern matching on restricted ADTs: we just assume they
are defined by a constructor and destructor.

Why are restricted data types sufficient?  Really, we don't
care about \emph{algebraic} data types, but the fact that
$\mathbf{data}$ declarations in Haskell are \emph{generative},
a problem we must tackle in this semantics.

\paragraph{What happened to export lists?}
In ordinary Haskell, users write import and export lists to help
guide the renaming process.  After renaming, imports and exports
are no longer necessary and can be dropped from the renamed intermediate
representation.  This can be most clearly seen by the following example:

\begin{verbatim}
    module M(x) where
        data T = MkT { unMkT :: Int }
        x = MkT 2
\end{verbatim}
%
This is a \emph{valid} module.  Furthermore, \verb|M.x| is a valid
identifier, despite \verb|T| being ``unmentionable''.  In actuality,
after renaming \verb|T| is part of the module type for \verb|M|, just
like \verb|x|; it is only the case that there isn't a way to specify
\verb|T| in the \emph{pre-renamed syntax}. As Explicit Backpack is
post-renaming, \verb|M| with and without the export list are represented
identically.

There is one catch, however: in Haskell, a module can \emph{reexport}
an identity from a different module:

\begin{verbatim}
    module A(x) where
        x = True
    module B(x) where
        import A (x)
\end{verbatim}

In Backpack, \verb|B| provides a type \verb|x|, and should be considered
to match a signature which contains \verb|x :: Bool|.  In POPL'14 and
our ICFP'16 submission, this matching is done by looking at Haskell's
module exports.  In Explicit Backpack, we will take a different
approach: If a signature reexports a type, or a module reexports a
type/value, we simply include an administrative binding in the module
$\mathbf{type}~n = \RN$ or $n = \RN$.  Thus, in Explicit Backpack,
these modules would be written as:
\[
\begin{array}{l}
\mathtt{A} = \{ \mathtt{x} = \mathtt{True} \} \\
\mathtt{B} = \{ \mathtt{x} = \mathtt{A}.\mathtt{x} \}
\end{array}
\]


\paragraph{Haskell types and expressions}
Here is our renamed Haskell-level types and expressions
(for simplicity, we assume all Haskell types have kind $\star$):

\begin{figure}[H]
\[
\begin{array}{rcll}
\RT &::=& & \mbox{Renamed Haskell types} \\
     & | & \RT \rightarrow \RT & \quad\mbox{Function type} \\
     & | & \RN & \quad\mbox{Type projection} \\
     & | & \cdots \\
\RE &::=& & \mbox{Renamed Haskell expressions} \\
  & | & \RE_1~\RE_2 & \quad\mbox{Function application} \\
  & | & \RN & \quad\mbox{Value projection} \\
  & | & \cdots & \\
\end{array}
\]
\caption{Renamed Haskell types and expressions}
\end{figure}

\paragraph{Paths}
To encode references to values which are defined in the same
module, or modules defined in the same component, we introduce
a notion of a \emph{path}.

For someone familiar with how renaming works in GHC, the presence
of paths in Explicit Backpack, which is \emph{renamed}, may be something of a surprise.  When
GHC renames a Haskell program, module qualified identifiers are
resolved to \emph{original names}, which uniquely identify the type
or value in question, even for types and values defined in the
module \emph{being compiled}.  This is convenient for implementation,
but when writing declarative rules, this style requires you to know,
as part of the environment, the name of the module (and the component)
which you are currently typechecking.  In this presentation, we opt
for paths in Explicit Backpack because they simplify the definition of
our context.

\begin{figure}[H]
\[
\begin{array}{rcll}
\RN &::=&     & \mbox{Name paths} \\
  & | & \RM.n & \quad\mbox{External name} \\
  & | & n   & \quad\mbox{Module-local name} \\
\RM &::=& & \mbox{Module paths} \\
  & | & M & \quad\mbox{External module} \\
  & | & m & \quad\mbox{Component-local module} \\
\end{array}
\]
\caption{Paths in the surface language}
\end{figure}

\paragraph{Module and component identities}
A path may refer to an external component, in which case we may need to
\emph{instantiate} the component with implementations for its
requirements.  Put differently, we need to apply our applicative
functors (components)!  Because Explicit Backpack only supports
applicative functors, it is unnecessary to have any constructs for
applying a functor and then binding it, since the result will always
be the same if you re-apply it later.

We identify applicative functor applications using the language of
component identities and module identities from our ICFP'16 submission.
An instantiated component identity is like a functor application, except
that that arguments are explicitly labeled with the name of the requirement
they fill; a module identity projects out a module from an instantiated
component, or refers to a hole module in the context.

Note: the stratification of paths and identities means that you cannot
use a locally defined module to instantiate a component (since
identities can't refer to local paths, only local holes).  This
restriction will simplify our component language.\footnote{You
may be wondering where name holes \nhv{m.n} are: these name holes are
only relevant in the semantic objects associated with this syntax.}

\begin{figure}[H]
\[
\begin{array}{rcll}
p &   & \mbox{Component identity} \\
m &   & \mbox{Module name} \\
\\
P &::=& p[\overline{m = M}] & \mbox{Instantiated component identity} \\
M &::=& \Mod{P}{m} & \mbox{Module identity} \\
  & | & \hv{m} & \quad\mbox{Hole module} \\
\end{array}
\]
\caption{Component and module identities (as in ICFP'16 submission)}
\end{figure}

\paragraph{Example}
Here is the running example from our ICFP'16 submission, with
two adjustments: first, $\mathbf{data}$ declarations have rewritten
so that they fit our restricted syntax; second, the module
reexport \modname{Z} in \cidl{r} has been replaced with a fresh module
which reeexports all the entities from
\modname{Y}:

\begin{figure}[H]
\[
\begin{array}{l}
\cidl{base} = \{ \\
  \quad\quad \modname{W} = \{ \\
  \quad\quad\quad \mathbf{data}~\mathtt{I} = \mathtt{MkI}~\{ \mathtt{unMkI} :: \mathtt{I} \}; \\
  \quad\quad\quad \mathtt{f} = \lambda x.~\ldots \\
  \quad\quad \} \\
  \quad\} \\

\cidl{p} = \lambda [ \modname{A} : \{ \mathbf{data}~\mathtt{J} \},
                     \modname{B} : \{ \mathbf{data}~\mathtt{K} \} ].\, \{ \\
  \quad\quad \modname{Y} = \{ \mathbf{data}~\mathtt{L} = \mathtt{MkL}~\{ \mathtt{unMkL} :: (\hv{A}.\mathtt{J}, \hv{B}.\mathtt{K}) \} \\
  \quad \} \\

\cidl{q} = \lambda [
    \modname{A} : \{ \mathbf{data}~\mathtt{I} = \mathtt{MkI}~\{ \mathtt{unMkI} :: \mathtt{I} \}
  ].\, \{ \\
  \quad\quad \modname{X} = \{ \mathbf{data}~\mathtt{K} = \mathtt{MkK}~\{ \mathtt{unMkK} :: \hv{A}.\mathtt{I} \} \\
  \quad \} \\

\cidl{r} = \lambda [
    \modname{A} : \{
      \textbf{type}~\mathtt{I} = \MOD{base}{}{W}.\mathtt{I};\;
      \textbf{data}~\mathtt{J}
    \}
  ].\, \{ \\
  \quad\quad \modname{Z} = \{
      \mathbf{type}~\mathtt{L} = \MOD{p}{ \subst{A}{\hv{A}}, \substMOD{B}{q}{ \subst{A}{\hv{A}} }{X} }{Y}.\mathtt{L}; \\
      \quad\quad\quad\quad~\mathtt{MkL} = \MOD{p}{ \subst{A}{\hv{A}}, \substMOD{B}{q}{ \subst{A}{\hv{A}} }{X} }{Y}.\mathtt{MkL}; \\
      \quad\quad\quad\quad~\mathtt{unMkL} = \MOD{p}{ \subst{A}{\hv{A}}, \substMOD{B}{q}{ \subst{A}{\hv{A}} }{X} }{Y}.\mathtt{unMkL} \quad \} \\
  \quad \}
\end{array}
\]
\caption{Running example in Explicit Backpack}
\end{figure}

\section{The elaboration}

The lesson of F-ing modules is that ML modules are simply a mode of use
of universals and existentials.  Backpack is closely related to ML
modules: and can be explained in a similar fashion.
In this section, we give an F-ing semantics for Backpack, as opposed
to the ad hoc semantics from our ICFP'16 submission.

\paragraph{Semantic objects} We start by defining our semantic signatures
($\alpha$, $\beta$, $\CK$ and $\CT$ all denote variables):

\begin{figure}[H]
\[
\begin{array}{rcl}
\kappa &::=& \cdots \\
\tau, \nu &::=& \alpha ~|~ \beta ~|~ \CK ~|~ \CT \\
       & | & \cdots \\
\I{ty} &::=& [ = \tau : \kappa ] ~ \\
       & | & [ \tau ] \\
\sigma &::=& \{ \overline{n :: \I{ty}} \,|\, r \}\\
\Sigma &::=& \{ \overline{m : \sigma } \} \\
\Xi    &::=& \exists \overline{(m = \CK)}\, \overline{\CT}\, \overline{r}.\, \forall \overline{(m = \alpha)}\, \overline{\beta}\, \overline{r}.\,
              \Sigma_R \rightarrow \Sigma_P
%   \Xi    &::=& \exists (\rho_P : \{ \overline{m = \KM^i \rightarrow \KM} \} )\,
%                        \overline{(\beta_P : \star^j \times \KM \rightarrow \star)}.\,
%                \forall (\rho_R : \{ \overline{m = \KM}^i \})\,
%                        \overline{(\beta_R : \star)}^j. \\
%          &   & \qquad \Sigma_R \rightarrow \Sigma_P \\
\end{array}
\]
\caption{Semantic signatures}
\end{figure}

\noindent
For brevity, we reuse the identifiers $\Sigma$, $\sigma$, $\tau$ and $\I{ty}$.
Our general strategy is to model a module as a record containing entries
representing the types and values defined in the module.  There are a few
finer points worth discussing.

\paragraph{Atomic signatures}  Like in F-ing modules, we need a way of
encoding \emph{types} in our records, so that type matching occurs when
we instantiate a component.  We assume that there are $\mathrm{F}_\omega$
types for $[\tau]$ and $[= \tau : \kappa]$, denoting value and type declarations
respectively.  The atomic signature forms are syntax sugar for $\mathrm{F}_\omega$
types; we assume an encoding similar to the one in F-ing modules (but
otherwise the details of the encoding is irrelevant.)

\paragraph{Abstract types}
In the surface syntax, we write \textbf{data} declarations and then refer
to them using paths.  In our semantic signatures, these paths are eliminated
and replaced with existentially quantified type variables $\beta$.
For example, if a user writes $\mathbf{data}~\texttt{I}$ in a signature, they get
a semantic type:

\begin{figure}[H]
\[
\exists (\beta_I : \star).\, \{ \mathtt{I} :: [ = \beta_I ] \}
\]
\caption{Example typing of an abstract data type}
\end{figure}

\noindent
In an F-ing semantics, these quantifiers get lifted to the top-level of
the program (the existential turning into a universal when it moves
out of the contravariant function position).
We will discuss these types in more detail shortly.

\paragraph{Encoding restricted algebraic data types}  Conspicuously, \textbf{data} is missing
from the set of productions for \I{ty}.  If a user writes
$\mathbf{data}~\mathtt{I} = \mathtt{MkI}~\{ \mathtt{unMkI} :: () \}$,
in a signature, we will model it as the following semantic type:

\begin{figure}[H]
\[
\exists (\beta_I : \star).\,
\{
\mathtt{I} :: [= \beta_I],
\mathtt{MkI} :: [() \rightarrow \beta_I],
\mathtt{unMkI} :: [\beta_I \rightarrow ()]
\}
\]
\caption{Example signature of a (generative) algebraic data type}
\end{figure}

\noindent
How does the elaboration work when such a declaration is written
in a module?  Essentially, we want to define a trivial implementation
of the record and then \emph{seal} it, so that the implementation details
of how a data-type are encoded are hidden from users, except through the
constructor and destructor.

\paragraph{Existential and universal quantification of types}
Roughly speaking (considering only $\beta$s), a component type
has the form $\Xi ::= \exists \overline{\beta_P}.\, \forall \overline{\beta_R}.\, \Sigma_R \rightarrow \Sigma_P$, which is precisely the ordering of quantifiers necessary
for applicative functors, as observed by Russo.

What are the kinds of these $\beta$?  For universally quantified types,
clearly $\beta : \star$ for any types which
had kind $\star$ in the source language.  But for existentially quantified
types, its abstract
type $\beta$ must be \emph{skolemized} over the universally
quantified types from the signature (as previously observed in Biswas
(1995), Russo (1998), and F-ing modules).  For example, consider
\texttt{L} from \cidl{p} in the running example:  considering
only types for the moment, we might naively write the following type for \cidl{p}:

\begin{figure}[H]
\[
\begin{array}{l}
\cidl{p} : \exists (\beta_L : \highlight{\star}).\, \forall (\beta_J\, \beta_K : \star).\, \\
    \quad \{
     \modname{A} :
        \{ \mathtt{J} :: [= \beta_J : \star] \},
     \modname{B} :
        \{ \mathtt{K} :: [= \beta_K : \star] \},
    \}
    \rightarrow \\
    \quad \{
     \modname{Y} :
        \{ \mathtt{L} :: [= \highlight{\beta_L} : \star ]
         ; \mathtt{MkL} :: (\beta_J, \beta_K) \rightarrow \highlight{\beta_L}
         ; \mathtt{unMkL} :: \highlight{\beta_L} \rightarrow (\beta_J, \beta_K)

         \}
    \}
\end{array}
\]
\caption{\emph{Incorrect} typing of \cidl{p}}
\end{figure}

\noindent
But in this signature, we have claimed that we can commit to
a representation of $\beta_L$, without knowing what $\beta_J$
or $\beta_K$ are.  This obviously is not going to work,
because we clearly want to define $\beta_L = (\beta_J, \beta_K)$
inside the sealed package representing an ADT\@.  So instead, $\beta_L$
must take these $\beta$s as a parameter.  To make the difference
visually clear, for existentially quantified variables we use
$\CT_L$ instead:

\begin{figure}[H]
\[
\begin{array}{l}
\cidl{p} : \exists (\CT_L : \highlight{\star^2 \rightarrow \star}).\, \forall (\beta_J\, \beta_K : \star).\, \\
    \quad \{
     \modname{A} :
        \{ \mathtt{J} :: [= \beta_J : \star] \},
     \modname{B} :
        \{ \mathtt{K} :: [= \beta_K : \star] \},
    \}
    \rightarrow \\
    \quad \{
     \modname{Y} :
        \{ \mathtt{L} :: [= \highlight{\CT_L\, \beta_J\, \beta_K} : \star]; \\
\qquad\quad~ \mathtt{MkL} :: (\beta_J, \beta_K) \rightarrow \highlight{\CT_L\, \beta_J\, \beta_K}; \\

\qquad\quad~ \mathtt{unMkL} :: \highlight{\CT_L\, \beta_J\, \beta_K} \rightarrow (\beta_J, \beta_K)
           \}
    \}
\end{array}
\]
\caption{Simplified typing of \cidl{p} (handling only $\beta$s)}
\end{figure}

\noindent
Now we have $\CT_L = \lambda \beta_J\, \beta_K.\, (\beta_J, \beta_K)$, as desired.
($\star^2 \rightarrow \star$ denotes $\star \rightarrow \star \rightarrow \star$.)

\paragraph{Labeled existential and universal quantification of module
identities}
The type identities $\CT$ described in the previous section are
incomplete, because they check for equivalence of the \emph{type
identities of the functional dependencies} rather than the \emph{module
identities} which provided these dependencies.  According to
Backpack'16, a component containing the declaration \verb|data I = MkI|
can be instantiated to form multiple distinct
type identities for \verb|I|, even though its definition doesn't mention
any types from any signature.  This problem is analogous to the
lack of ``dynamic abstraction safety'' described in the extended
F-ing modules paper.

The solution for this is to introduce a \emph{phantom type} to
help distinguish between these two types.  Intuitively, every type
identity is parametrized by a phantom type which encodes the
Backpack'16 module identity associated with this type.  Thus,
to refine the previous example further:

\begin{figure}[H]
\[
\begin{array}{l}
\cidl{p} : \exists \highlight{(\CK_Y : \KM^2 \rightarrow \KM)} (\CT_L : \highlight{\KM^2} \rightarrow \star^2 \rightarrow \star).\, \forall \highlight{(\alpha_A\, \alpha_B : \KM)} (\beta_J\, \beta_K : \star).\, \\
    \quad \{
     \modname{A} :
        \{ \mathtt{J} :: [ = \beta_J ] \},
     \modname{B} :
        \{ \mathtt{K} :: [ = \beta_K ] \},
    \}
    \rightarrow \\
    \quad \{
     \modname{Y} :
        \{ \mathtt{L} :: [ = \CT_L\,\highlight{\alpha_A\, \alpha_B}\, \beta_J\, \beta_K ]; \\
        \qquad\quad~ \mathtt{MkL} :: (\beta_J, \beta_K) \rightarrow \CT_L\,\highlight{\alpha_A\, \alpha_B}\, \beta_J\, \beta_K \}; \\
        \qquad\quad~ \mathtt{unMkL} :: \CT_L\, \highlight{\alpha_A\, \alpha_B}\, \beta_J\, \beta_K \rightarrow (\beta_J, \beta_K)  \}
    \}
\end{array}
\]
\caption{Typing of \cidl{p}, with rows and labels elided}
\end{figure}

\noindent
Ignoring the kinds for a moment, notice that each existentially
qualified $\CT$ now takes as arguments $\alpha_A\, \alpha_B$, which are
universally quantified over the \emph{module identities} which will
implement these signatures. The type arguments of $\CT$ are similar to
the structure of a module identity $\MOD{q}{\subst{A}{\hv{A}},
\subst{B}{\hv{B}}}{Y}$ (in this case, the existential $\CT$ encodes the
component identity, module name and type name, so only the module
identities of the holes are needed).

How are module identities constructed?  Every module defined in a
component comes with an existential $\CK$ types, e.g., $\CK_Y$, which represents
the module identity of the module in the component.  Notice that
$\CK_Y$ takes two arguments: we construct the final phantom type
by filling in these arguments with the phantom types of the modules that
$p$ is instantiated with.

In System $\mathrm{F}_\omega$, the only kind is $\star$;
$\CK_Y$ should have kind $\star^2 \rightarrow \star$; but to help distinguish
between $\beta$s and $\alpha$s, we define a kind synonym $\KM \equiv \star$,
denoting phantom types representing module identities; we use the metavariable
$\nu$ to refer to such phantom types.

There is one final technical complication: while we have treated
the arguments to $\alpha$ simply as an ordered list of phantom types,
it is instead convenient to consider them as module name labeled records, dictating \emph{which} argument
it is for, e.g., $\CK_Y\, \{ \modname{A} = \alpha_A, \modname{B} = \alpha_B \}$.

To accommodate this, in our full examples, we give $\CK$ the kind $\{\overline{m : \KM}\} \rightarrow \KM$,
and apply it by constructing a ``record'' of the form $\{\overline{m = \nu}\}$.
By fixing an ordering of $m$, these can be eliminated into usual, $n$-ary application.
Additionally, quantifiers for $\alpha$s take the form
$\exists (m = \alpha).\, {-}$, specifying that $\alpha$ binds the module
identity associated with the module named $m$.

\begin{figure}[H]
\[
\begin{array}{l}
\cidl{p} : \exists (\highlight{\modname{Y} =} \CK_Y : \highlight{\{\modname{A} : \KM, \modname{B} : \KM\}} \rightarrow \KM) (\CT_L : \highlight{\{\modname{A} : \KM, \modname{B} : \KM\}} \rightarrow \star^2 \rightarrow \star).\, \\
\quad~ \forall (\highlight{\modname{A} =} \alpha_A : \KM, \highlight{\modname{B} =} \alpha_B : \KM) (\beta_J\, \beta_K : \star).\, \\
    \quad \{
     \modname{A} :
        \{ \mathtt{J} :: [= \beta_J] \},
     \modname{B} :
        \{ \mathtt{K} :: [= \beta_K] \}
    \}
    \rightarrow \\
    \quad \{
     \modname{Y} :
        \{ \mathtt{L} :: [= \CT_L\, \highlight{\{\modname{A} = \alpha_A, \modname{B} = \alpha_B\}}\, \beta_J\, \beta_K \}]; \\
        \qquad\quad~\mathtt{MkL} :: (\beta_J, \beta_K) \rightarrow \CT_L\, \highlight{\{\modname{A} = \alpha_A\, \modname{B} = \alpha_B\}}\, \beta_J\, \beta_K; \\
        \qquad\quad~\mathtt{unMkL} :: \CT_L\,\highlight{\{\modname{A} = \alpha_A\, \modname{B} = \alpha_B\}}\, \beta_J\, \beta_K  \rightarrow (\beta_J, \beta_K)
        \quad \}
    \}
\end{array}
\]
\caption{Typing of \cidl{p}, with rows elided}
\end{figure}

\noindent
To summarize, the encoding scheme is as follows:

\begin{itemize}
    \item Every source module $m_P$ in some component $p$ is associated
    with an existentially quantified $\CK : \{\overline{m_R : \KM}\} \rightarrow \KM$,
    where $m_R$ ranges over the requirements of $p$.
    \item The phantom type representing a module identity $\MOD{p}{\overline{m = M}}{m'}$
    is found by recursively finding the phantom types of $\overline{M}$, and
    then applying these as arguments to the $\CK$ associated with $m'$ in component $p$.
    \item Every existentially bound type is skolemized
    over its type variables ($\tau$) as well as the phantom types ($\nu$)
    representing the module identity of the modules that fill
    the holes.
\end{itemize}

\paragraph{Existential and universal quantification of rows}
Finally, we quantify over \emph{row variables}. A universally
quantified row variable matches a module
that declares more fields than the module needs. (Technically,
the syntax permits
existentially quantified row variable, but we don't use this facility.)

\begin{figure}[H]
\[
\begin{array}{l}
\cidl{p} : \exists (\CT_L : \star \rightarrow \star \rightarrow \star).\, \forall (\beta_J\, \beta_K : \star)\, \highlight{(r_A\, r_B : \row)}.\, \\
    \quad \{
     \modname{A} :
        \{ \mathtt{J} :: \mathbf{type}~{-} = \beta_J \,|\, \highlight{r_A} \},
     \modname{B} :
        \{ \mathtt{K} :: \mathbf{type}~{-} = \beta_K \,|\, \highlight{r_B} \},
    \}
    \rightarrow \\
    \quad \{
     \modname{Y} :
        \{ \mathtt{L} :: \mathbf{type}~{-} = \CT_L\, \beta_J\, \beta_K~\mathbf{where}~\mathtt{MkL} :: \beta_J \rightarrow \beta_K \rightarrow \CT_L\, \beta_J\, \beta_K \}
    \}
\end{array}
\]
\caption{Simplified typing of \cidl{p} handling only $\beta$s with rows}
\end{figure}

\noindent
Row polymorphism obviates the need for a subtyping relation:
after picking an appropriate instantiation of $r_A$ and $r_B$,
we will have exact equality.

\paragraph{Example semantic typing context}
Here is a context which contains all of the components of our running example.

\begin{figure}[H]
\[
\begin{array}{rcl}
%\Mod{\cidl{base}}{W} \Rightarrow \CK_W : \KM; \\
\CK_W &:& \KM \\
\CT_I &:& \star \\
\cidl{base} &:& \{ \modname{W} : \{ \\
&&\quad \mathtt{I} :: [= \CT_I]; \\
&&\quad \mathtt{MkI} :: \CT_I \rightarrow \CT_I; \\
&&\quad \mathtt{unMkI} :: \CT_I \rightarrow \CT_I; \\
&&\quad \mathtt{f} :: \CT_I \rightarrow \CT_I \\
&&\} \} \\
\\
%\Mod{\cidl{p}}{Y} \Rightarrow \CK_Y : [ \modname{A} : \KM, \modname{B} : \KM ] \rightarrow \KM; \\
\CK_Y &:& \{ \modname{A} : \KM, \modname{B} : \KM \} \rightarrow \KM \\
\CT_L &:& \{ \modname{A} : \KM, \modname{B} : \KM \} \rightarrow \star^2 \rightarrow \star \\
\cidl{p} &:& \forall (\modname{A} = \alpha_A, \modname{B} = \alpha_B : \KM)\, (\beta_J, \beta_K : \star)\, (r_A\, r_B : \row). \, \{ \\
&& \quad\modname{A} : \{ \mathtt{J} :: [= \beta_J] \,|\, r_A \}, \\
&& \quad\modname{B} : \{ \mathtt{K} :: [= \beta_K] \,|\, r_B \} \\
&& \} \rightarrow \{ \\
&& \quad\modname{Y} : \{ \mathtt{L} :: [= \CT_L~ \{\modname{A} = \alpha_A, \modname{B} = \alpha_B\}\, \beta_J\, \beta_K]; \\
&& \quad\quad~~\texttt{MkL} :: (\beta_J, \beta_K) \rightarrow \CT_L~ \{\modname{A} = \alpha_A, \modname{B} = \alpha_B\}\, \beta_J\, \beta_K  \}; \\
&& \quad\quad~~\texttt{unMkL} :: \CT_L~\{\modname{A} = \alpha_A, \modname{B} = \alpha_B\}\, \beta_J\, \beta_K  \rightarrow (\beta_J, \beta_K)   \} \\
&& \} \\
\\
\CK_X &:& \{ \modname{A} : \KM \} \rightarrow \KM \\
\CT_K &:& \{ \modname{A} : \KM \} \rightarrow \star \rightarrow \star \\
\cidl{q} &:& \forall (\modname{A} = \alpha_A : \KM)\, (\beta_I : \star)\, (r_A : \row). \, \{ \\
&& \quad\modname{A} : \{ \mathtt{I} :: [= \beta_I]; \\
&& \quad\quad~~\mathtt{MkI} :: \beta_I \rightarrow \beta_I; \\
&& \quad\quad~~\mathtt{unMkI} :: \beta_I \rightarrow \beta_I \,|\, r_A \} \\
&& \} \rightarrow \{ \\
&& \quad\modname{X} : \{ \mathtt{K} :: [= \CT_K~ \{\modname{A} = \alpha_A\}\, \beta_I] \\
&& \quad\quad~~\mathtt{MkK} :: \beta_I \rightarrow \CT_K~\{\modname{A} = \alpha_A\}\, \beta_I \\
&& \quad\quad~~\mathtt{unMkK} :: \CT_K~\{\modname{A} = \alpha_A\}\, \beta_I \rightarrow \beta_I \}\\
&& \} \\
\\
\CK_Z &:& \{ \modname{A} : \KM \} \rightarrow \KM \\
\cidl{r} &:& \forall (\modname{A} = \alpha_A : \KM)\, (\beta_J : \star)\, (r_A : \row). \, \{ \\
&& \quad\modname{A} : \{ \\
&& \quad\quad \mathtt{I} :: [= \CT_I]; \\
&& \quad\quad \mathtt{J} :: [= \beta_J] \,|\, r_A \} \\
&& \} \rightarrow \{ \\
&& \quad\modname{Z} : \{ \mathtt{L} :: [= \CT_L~\{\modname{A} = \alpha_A, \modname{B} = \CK_X \{\modname{A} = \alpha_A\}\} \\
&& \qquad\qquad\qquad\qquad \beta_J \\
&& \qquad\qquad\qquad\qquad (\CT_K~\{\modname{A} = \alpha_A\}~\CT_I) \} \quad ]; \\
&& \qquad\qquad \ldots \\
&& \} \\
\end{array}
\]
\caption{Example typing context}
\end{figure}

\paragraph{The rules}

Here is the type system for this language.  Here are all the judgments
we must specify:

\begin{figure}[H]
\[
\begin{array}{ll}
\mbox{Type elaboration} &\Gamma \vdash T : \star \leadsto \tau \\
\mbox{Expression typing} & \Gamma \vdash E : \tau \\
\mbox{Component identity typing} & \Gamma \vdash P : \Sigma \Rightarrow [\overline{m = \nu}] \\
\mbox{Module identity typing} & \Gamma \vdash M : \sigma \Rightarrow \nu \\
\mbox{Module path typing} & \Gamma \vdash \RM : \sigma \\
\mbox{Name path typing} & \Gamma \vdash \RN : \tau \\
\mbox{Name path elaboration (for types)} & \Gamma \vdash \RN \leadsto \tau \\
\mbox{Signature binding elaboration} & \Gamma \vdash \I{rsigs} \leadsto \exists \overline{\alpha}\, \overline{\beta}\, \overline{r}.\, \Sigma \\
\mbox{Signature elaboration} & \Gamma \vdash \I{rsig} \leadsto \exists \overline{\beta}\, r.\, \sigma \\
\mbox{Signature declaration elaboration} & \Gamma \vdash \I{rsigd} \leadsto \exists \beta.\, \I{\sigma} \\
\mbox{Module binding typing} & \Gamma \vdash \I{rmods} :_P \exists \overline{(m = \CK)}\, \overline{\CT}.\, \Sigma \\
\mbox{Module typing} & \Gamma \vdash \I{rmod} :_P \exists \overline{\CT}.\, \sigma \\
\mbox{Module declaration typing} & \Gamma \vdash \I{rmodd} :_P \exists \CT.\, \I{\sigma} \\
\mbox{Component typing} & \Gamma \vdash \I{rcomp} :_P \Xi \\
\mbox{Program typing} & \Gamma \vdash \I{rprog} \\
\end{array}
\]
\caption{Judgments for typing}
\end{figure}

\newcommand{\tyenv}{\Gamma}

\noindent
Like in F-ing module's treatment of applicative modules,
$\Gamma \vdash M : \exists \overline{\alpha}.\, \tau \leadsto e$
actually means that $e$ is a closed term satisfying
$\cdot \vdash e : \exists \overline{\alpha}.\, \forall \tyenv.\, e $,
subject to the $\Gamma$ translation rules described in F-ing modules
(they are ``what you'd expect''.) \Red{A major question is whether or
not there is a more elegant way to set this up when you don't
have to deal with generative functors.}

Let us begin.\\

\begin{figure}[H]
\fbox{
$\Gamma \vdash \I{rprog}$
}

\[
\Gamma \vdash \varepsilon \elab{\{\}}
\]

\[
\frac{
\begin{array}{c}
\Gamma \vdash \I{rcomp} :_P \exists \overline{(m_P = \CK)}\, \overline{\CT}.\, \forall \overline{\alpha}\, \overline{\beta}\, \overline{r}.\, \Sigma_R \rightarrow \Sigma_P \elab{e} \\
\Gamma, \overline{\Mod{p}{m_P} \Rightarrow \CK}, \overline{\CT}, p : \forall \overline{\alpha}\, \overline{\beta}\, \overline{r}.\, \Sigma_R \rightarrow \Sigma_P \vdash \I{rprog} \elab{e'}
\end{array}
}{
\Gamma \vdash p = \I{rcomp}; \I{rprog} \elab{\lambda \Gamma.\, \textsf{unpack}~\langle\overline{\CK}\, \overline{\CT}, p \rangle = e~\textsf{in}~e'~(\Gamma, p~\Gamma : \forall \overline{\alpha}\, \overline{\beta}\, \overline{r}.\, \Sigma_R \rightarrow \Sigma_P)}
}
\]
Given the type of a component, we unpack the existentials into the
context and continue typechecking the rest of the component.  Note
that the $\CK$ variables are labeled with $\Mod{p}{m_P}$ indicating
what module they correspond to; the identity typing rules will
take advantage of this information.  Since the labeling of $\CK$
is just bookkeeping information for the elaboration, we elide it
from the actual elaborated terms.
\caption{Program typing}
\end{figure}

\begin{figure}[H]
\fbox{
$\Gamma \vdash \I{rcomp} :_P \Xi \elab{e}$
}

\[
\frac{
\begin{array}{c}
\Gamma \vdash \I{rsigs} \leadsto \exists \overline{(m_R = \alpha)}\, \overline{\beta}\, \overline{r}.\, \Sigma_R \\
\Gamma, \overline{(m_R \Rightarrow \alpha)}, \overline{\beta}, \overline{r}, \Sigma_R \vdash \I{rmods} :_P \exists \overline{(m_P = \CK)}\, \overline{\CT}.\, \Sigma_P \elab{e} \\
%\kappa_{\CK} = [\overline{m_R : \KM}] \rightarrow \KM \\
%\kappa_{\CT} = \KM \rightarrow \overline{\star}^R \rightarrow \star \\
\end{array}
}{
\Gamma \vdash \lambda [ \I{rsigs} ].\, \{ \I{rmods} \} :_P
\exists \overline{(m_P = \CK)}\, \overline{\CT}.\,
\forall \overline{(m_R = \alpha)}\, \overline{\beta}\, \overline{r}.\,
\Sigma_R \rightarrow \Sigma_P
\elab{e}
}
\]
The component typing rule is simply elaborates the signature,
and then typechecks the module bodies with the signature in the
context.  For technical reasons, the module declaration returns
the \emph{already skolemized} type variables.
\caption{Component typing}
\end{figure}

\begin{figure}[H]
\fbox{
$\Gamma \vdash \I{rsigs} \leadsto \exists \overline{\alpha}\, \overline{\beta}\, \overline{r}.\, \Sigma$
}

\[
\Gamma \vdash \varepsilon \leadsto \{ \}
\]

\[
\frac{
\begin{array}{c}
\alpha~\mbox{fresh}\\
\Gamma \vdash \I{rsig} \leadsto \exists \overline{\beta}\, r.\, \sigma \\
\Gamma, m \Rightarrow \alpha, \overline{\beta}, r, m : \sigma \vdash \I{rsigs} \leadsto \exists \overline{m' = \alpha'}\, \overline{\beta'}\, \overline{r'}.\, \{\overline{m' : \sigma'}\} \\
\end{array}
}{
\Gamma \vdash (m : \{ \I{rsig} \}; \I{rsigs}) \leadsto \exists (m = \alpha)\, \overline{(m' = \alpha')}\, \overline{\beta}\, \overline{\beta'}\, r\, \overline{r'}.\, \{ m : \sigma, \overline{m' : \sigma'} \}
}
\]
Lifts out existentials, allocating a fresh $\alpha$ to represent
the module identity of the unknown module implementing a signature.
\caption{Signature binding elaboration}
\end{figure}

\begin{figure}[H]
\fbox{
$\Gamma \vdash \I{rsig} \leadsto \exists \overline{\beta}\, r.\, \sigma$
}

\[
\Gamma \vdash \varepsilon \leadsto \exists r.\, \record{r}
\]

\[
\frac{
\begin{array}{c}
\Gamma \vdash \I{rsigd} \leadsto \exists \overline{\beta}.\, \{ \overline{n :: \I{ty}} \} \\
\Gamma, \overline{\beta}, \overline{n :: \I{ty}} \vdash \I{rsig} \leadsto \exists \overline{\beta'}\, r.\, \{ \overline{n' :: \I{ty}'} \} \\
\end{array}
}{
\Gamma \vdash (\I{rsigd}; \I{rsig}) \leadsto \exists \overline{\beta}\, \overline{\beta'}\, r.\, \{ \overline{n :: \I{ty}}, \overline{n' :: \I{ty}'} \}
}
\]
\caption{Signature elaboration}
\end{figure}


\begin{figure}[H]
\fbox{$\Gamma \vdash \I{rsigd} \leadsto \exists \overline{\beta}.\, \I{\sigma}$}

\[
\Gamma \vdash \mathbf{data}~n \leadsto \exists \beta.\, \{ n :: [= \beta] \}
\]

\[
\frac{
\Gamma \vdash T : \star \leadsto \tau
}{
\Gamma \vdash \mathbf{data}~n = n_c~\{ n_d :: T \}
    \leadsto
\exists \beta.\,
\{ n :: [=\beta], n_c :: [\tau \rightarrow \beta], n_d :: [\beta \rightarrow \tau] \}
}
\]

\[
\frac{
\Gamma \vdash T : \star \leadsto \tau
}{
\Gamma \vdash \mathbf{type}~n = T \leadsto \{ n :: [= \tau] \}
}
\]

\[
\frac{
\Gamma \vdash T : \star \leadsto \tau
}{
\Gamma \vdash n :: T \leadsto \{ n :: [\tau] \}
}
\]
\caption{Signature declaration elaboration}
\end{figure}


\begin{figure}[H]
\fbox{$\Gamma \vdash \RN \leadsto \tau$}

\[
\frac{
\Gamma(n) = [= \tau]
}{
\Gamma \vdash n \leadsto \tau
}
\]

\[
\frac{
\begin{array}{c}
\Gamma \vdash \RM : \sigma \\
\sigma(n) = [= \tau]
\end{array}
}{
\Gamma \vdash \RM.n \leadsto \tau
}
\]
\caption{Name path elaboration (for types)}
\end{figure}

\begin{figure}[H]
\fbox{$\Gamma \vdash \RN : \tau \elab{e}$}

\[
\frac{
\Gamma(n) = \tau
}{
\Gamma \vdash n : \tau \elab{n}
}
\]

\[
\frac{
\begin{array}{c}
\Gamma \vdash \RM : \sigma \elab{e} \\
\sigma(n) = \tau
\end{array}
}{
\Gamma \vdash \RM.n : \tau \elab{e.n}
}
\]
\caption{Name path typing}
\end{figure}

\begin{figure}[H]
\fbox{$\Gamma \vdash \RM : \sigma \elab{e}$}

\[
\frac{
\Gamma(m) = \sigma
}{
\Gamma \vdash m : \sigma \elab{m}
}
\]

\[
\frac{
\Gamma \vdash M : \sigma \Rightarrow \nu \elab{e}
}{
\Gamma \vdash M : \sigma \elab{e}
}
\]

\caption{Module path typing}
\end{figure}

\begin{figure}[H]
\fbox{$\Gamma \vdash M : \sigma \Rightarrow \nu \elab{e}$}

\[
\frac{
\begin{array}{c}
\Gamma \vdash P : \Sigma \Rightarrow [m = \nu, \ldots] \elab{e} \\
\Sigma(m) = \sigma
\end{array}
}{
\Gamma \vdash \Mod{P}{m} : \sigma \Rightarrow \nu \elab{e.m}
}
\]

\[
\frac{
\begin{array}{c}
\Gamma(m) \Rightarrow \alpha \\
\Gamma(m) = \sigma
\end{array}
}{
\Gamma \vdash \hv{m} : \sigma \Rightarrow \alpha \elab{m}
}
\]
This judgment also computes the phantom type variable
which represents the type in question.

\caption{Module identity typing}
\end{figure}

\begin{figure}[H]
\fbox{$\Gamma \vdash P : \Sigma \Rightarrow [\overline{m = \nu}] \elab{e}$}

\[
\frac{
\begin{array}{c}
\Gamma(p) = \forall \overline{(m_i = \alpha_i)}\, \overline{\beta}\, \overline{r}.\, \Sigma_R \rightarrow \Sigma_P \\
\overline{\Gamma(p.m_j) \Rightarrow \CK_j} \\
\forall i.\quad \Gamma \vdash M_i : \sigma_i \Rightarrow \nu_i \elab{e_i} \\
\Gamma \vdash \{ \overline{m_i : \sigma_i} \} = \exists \overline{\beta}\, \overline{r}.\, \substw{\Sigma_R}{\overline{\alpha_i \mapsto \nu_i}} \uparrow \overline{\tau}, \overline{c} \\
\end{array}
}{
\Gamma \vdash p[\overline{m_i = M_i}] : \substw{\Sigma_P}{\overline{\alpha_i \mapsto \nu_i}, \overline{\beta \mapsto \tau}} \Rightarrow [\overline{m_j = \CK_j~\overline{\nu_i}}] \elab{ p~\overline{\alpha_i = \nu_i}~\overline{\tau}~\overline{c}~\{\overline{m_i = e_i} \} }
}
\]

\caption{Instantiated component identity typing}
\end{figure}

\begin{figure}[H]
\fbox{$\Gamma \vdash \Sigma = \exists \overline{\beta}\, \overline{r}.\, \Sigma' \uparrow \overline{\tau}, \overline{c} $}

\[
\frac{
\Sigma = \substw{\Sigma'}{\overline{\beta \mapsto \tau}, \overline{r \mapsto c}}
}{
\Gamma \vdash \Sigma = \exists \overline{\beta}\, \overline{r}.\, \Sigma' \uparrow \overline{\tau}, \overline{c}
}
\]

Critically, every $\beta$ has kind $\star$ (or, if it has
a higher kind, it must unify with a variable of the same kind,
as Haskell lacks type-level lambda).  Thus, $\beta$ and $r$ can
be solved via ordinary unification.

(In contrast, F-ing modules had a complex notion of \emph{explicitness}
and \emph{validity}. We do not need these features, as we don't
support functor signatures; however, a simplified form of \emph{rootedness}
does hold for our signatures.)

\caption{Signature matching}
\end{figure}


\begin{figure}[H]
\fbox{$\Gamma \vdash \I{rmods} :_P \exists \overline{(m = \CK)}\, \overline{\CT}.\, \Sigma \elab{e}$}

\[
\Gamma \vdash \varepsilon :_P \{ \} \elab{ \{\}}
\]

\[
\frac{
\begin{array}{c}
\overline{\Gamma(m_R) \Rightarrow \alpha} \\
\CK : [\overline{m_R : \KM}] \rightarrow \KM~\mbox{fresh}\\
\Gamma \vdash \I{rmod} :_P \exists \overline{\CT}.\, \sigma \elab{e_1} \\
\Gamma, \overline{\CT}, m : \sigma \vdash \I{rmods} :_P \exists \overline{(m' = \CK')}\, \overline{\CT'}.\, \Sigma' \elab{e_2} \\
\end{array}
}{
\begin{array}{c}
\Gamma \vdash (m : \{ \I{rsig} \}; \I{rsigs}) :_P \exists (m = \CK)\, \overline{(m' = \CK')}\, \overline{\CT}\, \overline{\CT'}.\, \{ m : \sigma \,|\, \Sigma' \} \\
\elab{
\begin{array}{l}
\textsf{unpack}~\langle \overline{\CT}, y_1 \rangle = e_1~\textsf{in} \\
\textsf{unpack}~\langle \overline{(m' = \CK')}\, \overline{\CT'}, y_2 \rangle = e_2~\textsf{in} \\
\textsf{pack}~\langle \overline{(m = \lambda \overline{\nu}.\, \{\})\, (m' = \CK')}\, \overline{\CT}\, \overline{\CT'}, \lambda \Gamma.\, \{ m = y_1~\Gamma \,|\, y_2~(\Gamma, \overline{\CT}, y_1\,\Gamma : \sigma) \} \rangle
\end{array}
}
\end{array}
}
\]

The witness type for the phantom $\CK$ doesn't matter so we just pick $\lambda \overline{\nu}.\, \{\}$.  Note we don't put the fresh $\CK$ in the context as it is impossible for it to be used.

\caption{Module binding typing}
\end{figure}

\begin{figure}[H]
\fbox{
$\Gamma \vdash \I{rmod} :_P \exists \overline{\CT}.\, \sigma \elab{e}$
}

\[
\Gamma \vdash \varepsilon :_P \{ \} \elab{\lambda\tyenv.\, \{\}}
\]

\[
\frac{
\begin{array}{c}
\Gamma \vdash \I{rmodd} :_P \exists \overline{\CT}.\, \{ \overline{n :: \I{ty}} \} \elab{e_1} \\
\Gamma, \overline{\CT}, \overline{n :: \I{ty}} \vdash \I{rmod} :_P \exists \overline{\CT'}. \{ \overline{n' :: \I{ty}'} \} \elab{e_2} \\
\end{array}
}{
\Gamma \vdash (n :: \I{rmodd} ; \I{rmod}) :_P \exists \CT\, \overline{\CT'}.\, \{ \overline{n' :: \I{ty}'}, \overline{n :: \I{ty}} \} \elab{\begin{array}{l}
\textsf{unpack}~\langle \overline{\CT}, y_1 \rangle = e_1~\textsf{in}\\
\textsf{unpack}~\langle \overline{\CT'}, y_2 \rangle = e_2~\textsf{in}\\
\textsf{pack}~\langle \overline{\CT}\, \overline{\CT'},\\
\quad\lambda \Gamma.\, \textsf{let}~\overline{n = (y_1~\Gamma).n}~\textsf{in} \\
\quad\quad \{ \overline{n' = (y_2~(\Gamma, \overline{\CT}, \overline{n :: \I{ty}})).n'}, \overline{n = n} \} \rangle
\end{array}}
}
\]

\caption{Module typing}
\end{figure}

\begin{figure}[H]
\fbox{$\Gamma \vdash \I{rmodd} :_P \exists \overline{\CT}.\, \I{\sigma}$}

\[
\frac{
\begin{array}{c}
\Gamma \vdash T : \star \leadsto \tau \\
\end{array}
}{
\begin{array}{c}
\Gamma \vdash (\mathbf{data}~n = n_c~\{ n_d :: T \})
    :_P
\exists \CT.\,
\{ n :: [ = \CT~\Gamma]; n_c :: [\tau \rightarrow \CT~\Gamma]; n_d :: [\CT~\Gamma \rightarrow \tau] \} \\
\elab{
\textsf{pack}~\langle
\lambda \tyenv.\, \tau,
\lambda \tyenv.\, \{n = [\tau], n_c = \lambda x : \tau.\, x, n_d = \lambda x : \tau.\, x\} \rangle
}
\end{array}
}
\]

\[
\frac{
\Gamma \vdash T : \star \leadsto \tau
}{
\Gamma \vdash (\mathbf{type}~n = T) :_P \{ n :: [ = \tau ] \}
\elab{\lambda \tyenv.\, \{ n = [\tau] \}}
}
\]

\[
\frac{
\Gamma \vdash E :: \tau \elab{e}
}{
\Gamma \vdash n = E :_P \{ n :: \tau \} \elab{\lambda \tyenv.\, \{ n = [e] \}}
}
\]
\caption{Module declaration typing}
\end{figure}

\section{Soundness}

We can now show soundness of the F-ing semantics by showing that the
elabaration only produces well-typed terms of the target language.

\begin{theorem}[Soundness of elaboration]
We have:
\begin{enumerate}
    \item If $\Gamma \vdash T/\RN : \star \leadsto \tau$, then $\Gamma \vdash \tau : \star$.
    \item If $\Gamma \vdash E/\RN : \tau \leadsto e$, then $\Gamma \vdash e : \tau$.
    \item If $\Gamma \vdash P : \Sigma \leadsto e$, then $\Gamma \vdash e : \Sigma$.
    \item If $\Gamma \vdash M/\RM : \sigma \leadsto e$, then $\Gamma \vdash e : \sigma$.
    \item If $\Gamma \vdash \I{rsigs}/\I{rsig}/\I{rsigd} \leadsto \xi$, then $\Gamma \vdash \xi : \star$.
    \item If $\Gamma \vdash \I{rmods} :_P \exists \overline{\CK}\, \overline{\CT}.\, \Sigma \leadsto e$, then $\cdot \vdash e : \exists \overline{\CK}\, \overline{\CT}.\, \forall \Gamma.\, \Sigma$.
    \item If $\Gamma \vdash \I{rmod} :_P \exists \overline{\CT}.\, \sigma \leadsto e$, then $\cdot \vdash e : \exists \overline{\CT}.\, \forall \Gamma.\, \sigma$.
    \item If $\Gamma \vdash \I{rmodd} :_P \exists \overline{\CT}.\, \sigma \leadsto e$, then $\cdot \vdash e : \exists \overline{\CT}.\, \forall \Gamma.\, \sigma$.
    \item If $\Gamma \vdash \I{rcomp} :_P \exists \overline{\CK}\, \overline{\CT}.\, \xi \leadsto e$, then $\cdot \vdash e : \exists \overline{\CK}\, \overline{\CT}.\, \forall \Gamma.\, \xi$.
\end{enumerate}
\end{theorem}

This combined with soundness of the target language establishes soundness of the
entire language.

\end{document}
