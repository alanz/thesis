%!TEX root = paper.tex
\chapter{Related Work}

\paragraph{Comparison with \OldBackpack{}}  The inspiration for this
work was the original Backpack paper~\cite{backpack}.  Backpack was
first to pose the problem of retrofitting Haskell with interfaces, and
many of its design ideas, such as mixin packages, applicativity and
module identities have been preserved in this paper.  The contribution
of this paper is an actual \emph{implementation} of the
Backpack design, by refactoring of these ideas into a form that can be
implemented in two stages: mixin linking handled by the package manager,
and typechecking and compilation handled by the compiler.

This refactoring necessitated a change to one of the core definitions
in the package language: in \OldBackpack{}, type equality was based on
a per-module computed \emph{module identity}.  In effect, every
defined module separately kept track of the set of signatures that it
transitively imported.  If mixin linking is not allowed to inspect
source code, the notion of identity must be coarsened. In this paper,
\emph{instantiated component identities} track all of the requirements
in the component, which are computed during mixin linking.

A technical accomplishment of the original Backpack was a solution to
the ``double vision problem'', where the typechecker can see two
distinct names for the same underlying type in a mutually recursive
module.  In Backpack, this problem was solved by way of a ``shaping pass'' which
first computed the identities of what we call name variables prior to type
checking.  In this paper, we do not permit
mutual recursion; thus, the double vision problem does not occur,
and our ``shaping pass'' only computes a very coarse-grained shape
on the overall wiring structure of components.  From this wiring
structure, \Backpack{} can compute the correct identities for
name variables at the same time as type checking.

We have also taken a different approach to defining the semantics of
\Backpack{}.  Instead of elaborating to Haskell, which is not how you would ever
implement a Backpack-like system, we give an ad hoc semantics which
directly reflects how \Backpack{} has been implemented in GHC\@.

\paragraph{Modularity in the package manager}  While there is far more
literature on module systems that can be implemented entirely by a
compiler, there has been some work which has looked at the problem of
modular development at the package level.

One such system
is the Functoria DSL\footnote{\smaller \url{https://mirage.io/blog/introducing-functoria}}
of MirageOS~\cite{mirageos}.  MirageOS is a library operating system
written in OCaml, which provides modules and functors for constructing
unikernels.  Rather than manually instantiate these functors,
users write in the Functoria DSL, which describes what
dependencies to install (via the OCaml package manager) and how the ML
functors should be assembled.  Unlike \Backpack{}, their DSL follows
the model of explicit functor applications rather than mixin linking.

The Nix package manager~\cite{dolstra:thesis} is a
system for enabling reproducible builds of packages. Nix defines a (pure, functional) language of component
\emph{derivations}---\ie{} the source code and configuration
needed to build the derived component---%
functorized over configuration parameters and derivations of
depended-upon components.  Components are linked together with explicit
functor application, albeit with some of the syntactic convenience of
mixin linking.  However, there is no type system for components,
and thus the Nix output hashes (similar to our \cid{}s) only serve
the role of uniquely identifying derivations.

The SMLSC extension to Standard ML~\cite{swasey+:smlsc}, while
primarily intended as a mechanism to support separate compilation in
Standard ML, also has some similarities to \Backpack{}.  Like
Backpack, SMLSC operates at the level of \emph{units} (our
components), and defines interfaces between units to allow them to be
separately typechecked.  Unlike \Backpack{}, SMLSC does not support
reusing units with different
implementations of their interfaces: dependencies in SMLSC are always
\emph{definite references}, and signatures are used purely to permit
separate compilation.  In SMLSC, if you want multiple instantiations,
you are expected to use ML functors.

An unusual case of \emph{not} using a package manager when it would be
useful occurs in C++ templates.\footnote{\smaller
  \url{https://gcc.gnu.org/onlinedocs/gcc/Template-Instantiation.html}}
C++ templates are
applicative, in the sense that two occurrences of \verb|vector<int>| refer to the
same type.  However, the C++ compiler must generate code when a
template is instantiated. Implemented naively, this could result in a
lot of duplicate copies of code.  One early method of handling this
problem, ``Cfront model'', involved a template database where
instances of templates were maintained.  However, it was too
complicated for most C++ compilers to handle this database, and so the
usual ``Borland model'' (implemented by GCC, among others) is to just
recompile every template instantiation and deduplicate them at link
time.  With Backpack, we already have a package manager, Cabal, which
administers its own installed package database, so we can offload the
caching of instantiated components to it.  (This technique would not
work for C++ templates, whose type based dispatch must be deeply
integrated with the compiler.)

\paragraph{ML functors}  The original Backpack language distinguished
itself from ``functors'' in (variants of) the ML module system
\cite{milner+:def-of-sml-revised,ocaml} by the fact that it supported
separate type checking for recursive modules under applicative
instantiation.  Additionally, by being a mixin system, it is a
better fit for the package language and avoids the need for
sharing constraints.

As this paper does not address mutual recursion, one may wonder
if the \unit{} language is not simply just a stylized applicative
functor language. In fact, it is!  The reason our technical presentation
is done in the way it is done here is because our primary goal
was integrating with the existing compiler infrastructure.

\paragraph{Mixin linking} There is a rich literature in the mixin
linking world, which both Backpack and \Backpack{} draw heavily from
\cite{ancona+:cms,flatt+:units,rossberg+:mixml}.  Indeed,
the relationship to this literature is even clearer in \Backpack{}, as
the mixin linking step is factored out and is independent of the
Haskell language.  For example, the basic algorithm for linking in
Cardelli's \emph{linksets} calculus~\cite{cardelli:linksets}, at a
high level, is essentially the same algorithm as our mixin linking.
The difference, however, is that we must keep track of the structure
of the ``wiring diagram'', as this structure will be used to establish
the identities of types at the Haskell level.  In contrast, Cardelli
gave no account of the interaction between module-level linking and
core-level user-defined abstract data types.

The object oriented community has also studied mixin-style composition
in their designs.  However, these mechanisms are organized around
dynamic binding and objects; whereas in Backpack-style systems,
the emphasis is on packages.  Users of \Backpack{} pay no performance
penalty switching from a direct dependency to an indirect dependency
via a signature, because we don't do separate compilation of \Backpack{}.








%%% Local Variables:
%%% mode: latex
%%% TeX-master: "paper"
%%% End:
