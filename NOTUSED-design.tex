%!TEX root = paper.tex
\section{Design and implementation constraints}

\Red{This section is now somewhat duplicate with ``The problem'', I haven't
tried editing very much yet pending feedback on the approach of the problem.
One delicate thing is that we do want examples in this section, but we need
to have introduced the syntax first.}

The specific constraints facing the implementation of a mix-in,
applicative \emph{package} system for Haskell were non-obvious at
the beginning of this project.  In this section, we spend some time
explaining the constraints on the design which stemmed from Backpack's
focus on the package level rather than the module level.

\Red{This section should have forward references but they are missing
because the sections they should point to don't exist yet}

%   Backpack is a system for mix-in \emph{packages}, not modules, which puts
%   it in an unusual corner of the design space: it is as much an extension
%   to Cabal, Haskell's \emph{package manager}, as it is to GHC, in
%   contrast to most other module systems (ML modules, Scala mix-ins, etc.)
%   which are strictly compiler affairs.  This is good and bad: it means
%   that you can actually use Backpack to program against interfaces at the
%   package level, but, na\"\i vely, it also means you must interact with the
%   package manager to use Backpack.

%   This does not mean Backpack cannot be used to do ML-style modular
%   development at the level of modules rather packages.  But the primary
%   focus of Backpack on a package level rather than a module level
%   one leads to a number of interesting constraints on the design
%   and implementation of Backpack, which we will discuss in this section.

\section{Optimizing against dependencies}
\label{sec:optimizing}

\Red{For example, maybe can drop this section now.}

Today, when \verb|mylib| is built against \verb|mysql|, \verb|mylib|
takes advantage of having an implementation in hand to inline code from
\verb|mysql|.  This can help performance quite a bit, especially when
\verb|mysql| defines many small functions, for which inlining exposes
optimization opportunities.  Backpack should have similar performance
characteristics, which means that when we instantiate a package with
mix-in linking, we need to recompile it for the correct dependencies.
Compare this to Haskell's type classes: functions using a type class are
not specialized without requesting inlining.

\Scott{This seems like a much clearer explanation of what was described (and
what I commented on) in the ``problem'' section. If that section becomes more
about the diff between POPL backpack and ICFP backpack, then this bit here
seems clearer and better positioned.}

\section{Applicativity and the installed package database}

Suppose that you want to combine \verb|myapp| with another application
which also uses \verb|mylib|:

\begin{verbatim}
name: otherapp
version: 1.0
library
    build-depends: mylib == 2.*, mysql
    backpack-includes:
        mysql (MySQL as Database)
    exposed-modules: OtherApp

name: multiapp
version: 1.0
library
    build-depends: myapp, otherapp
    exposed-modules: MultiApp
\end{verbatim}

In both \verb|myapp| and \verb|otherapp|, \verb|mylib| has been
instantiated with a \verb|Database| implementation from \verb|mysql|.
Indeed, you might reasonably assume that the types and values from the
two instances \verb|mylib| ought to be exactly the same, much in the
same way that if we hadn't used Backpack signatures and instead gave
\verb|mylib| a direct dependency on \verb|mysql|, it would obviously be
shared.  In module systems, this property is called
\emph{applicativity}, which states that if two mix-in packages are
instantiated in the same way, they have exactly the same code and types.

The implication of sharing code and types this way is that, as we
are building a application we have to remember which packages we have
instantiated globally (since they must be shared.)  In fact, we can
do better: we can use an \emph{installed package database} to share
these instantiation across all builds.

\section{Instantiate packages, not modules}
\label{sec:instantiate-pkgs}

In the traditional Cabal model, dependencies declared for \verb|mylib|
affect all modules in \verb|mylib|, even if not all of the modules
actually (indirectly) depend on them.  Thus, the dependency structure of
packages is manifestly clear without having to look at Haskell source
code, and a package can be compiled as a whole into a library file
(e.g., an \verb|so| or \verb|a| file), which then can be linked in
statically or dynamically by existing operating system facilities.
These properties should be preserved by Backpack.

In fact, the original Backpack design failed this property: modules had
their dependencies on signatures tracked individually, which would have
made it very difficult to actually compile a package into a single
shared library.\footnote{The situation gets even worse if one suggests
that each declaration should be considered a module.}  Furthermore, you
could not tell what the requirements of a module were without inspecting
the import graph.

\section{Distinguish between packages and components}

In both traditional Cabal and Backpack, there are two primary
motivations behind packages.  First, the package is how code is
distributed: a package consists of a tarball with code and metadata
which is then uploaded to a package repository.  Second, packages track
dependencies: a package is the unit of code which another package can
depend on.

These motivations are sometimes in tension: for example, in Cabal today,
a user may want to \emph{distribute} a library and a test suite for the
library in the same package, but want to ascribe different
\emph{dependencies} two each of these components (the library does not
depend on a test harness).  Cabal deals with this tension by permitting
a package (the unit of distribution) to contain multiple components (the
unit of dependency), as in the following example, where an algorithm
library and its test suite are distributed together, but the test suite
has different dependencies:

\begin{verbatim}
name: algorithm
version: 1.0
library
    build-depends: base, containers
test-suite prop-compiled
    build-depends: algorithm, base, QuickCheck
\end{verbatim}

Backpack continues in this direction: Backpack is not a system for
mix-in packages but really a system for mix-in \emph{components}.  (It
just so happens that most packages contain a single component.) The use
of components allows us to use Backpack in cases where it would be
painful to make separate packages.  In the rest of this paper, we will
use the term \emph{component},  as it is the appropriate level of
granularity, and the problem packages solve (distribution) is out of
scope for this paper.

\section{Incremental compilation}

When we make a change to Haskell source code, GHC avoids recompiling
any modules which did not depend on that source code: an obviously
desirable feature!  However, incremental compilation interacts
poorly with certain features, e.g., a \emph{shaping pre-pass} which
processes every module first before actually type-checking.
\Scott{Here, and probably elsewhere, we need to more carefully separate
what's in the POPL system (eg shaping) and what's in the present system.}

% It is simple to dodge this bullet, however: shaping is only necessary in
% the presence of mutual recursion.

\section{Migration}

There is already a large amount of code published in the Haskell
ecosystem, and it would be unreasonable to expect everyone to switch
over to Backpack overnight.  Dependencies specified by Backpack
signatures (indirect dependencies) must coexist with traditional
\verb|build-depends| dependencies (direct dependencies, modulo
dependency solving).
