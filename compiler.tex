%!TEX root = paper.tex
\chapter{Type checking}
\label{sec:compiler}

We now give a formal model for how our implementation typechecks
\unit{}s which still have unfilled requirements.  It's not possible to
give a full semantics, since that would involve formalizing all of
Haskell as implemented by GHC, but we will informally explain the
operation of the judgments we assume are given to us.

\section{Preliminaries}

\paragraph{The context}  There is a bit of accounting to do:

\begin{itemize}
    \item The component type context $\Gamma ::= \overline{p : \Xi}$
    records the types of all components we have previously typechecked;
    these components are typed but not instantiated.  Type lookup will find a
    component type and the instantiate it to get an actual type.
    \item The module variable context $\Theta ::= \overline{\hv{m}}$
    specifies the set of module variables which are in scope.  This context is computed by mixin
    linking and recorded in the $\Theta$ in $\mcomp$.
    \item The name variable context $\theta ::= \overline{\nhv{m.n}}$
    specifies the set of name variables that are in scope.  Each should be
    ascribed a type
    in $\Sigma_R$ (described next.)
    \item The local requirements context $\Sigma_R ::= \overline{m : \Umty}$ specifies the
    module types of the requirements of the current component.  In this formal
    development, we assume that $\Sigma_R$ is given to us nondeterministically,
    see Section~\ref{sec:merging} for more details.
    \item The local provisions context $\Sigma_P ::= \overline{m : \Umty}$ specifies
    the module types of modules defined in the current component.
    Collectively, $\Delta ::= (\Sigma_R; \Sigma_P)$ is the full
    local context, which  will be transformed into the final type of the overall component.
    \item The import context $\impctx ::= \overline{\prov{m}{M}}$ maps module names to module
    identities, and specifies which module is actually imported by an
    \texttt{import} declaration in Haskell source, bringing the exports
    of that module into scope.
    \item The current \cid{} $p_0$ says what the current component is,
    and is used to generate local module identities and say when a type
    is local as opposed to global.
\end{itemize}

\begin{definition}[Name substitutions] \normalfont{}
Intuitively, name substitutions refine name variables to more specific
original names; in Section~\ref{sec:overview-compiler}, this operation
was shown diagrammatically by drawing a line between two entities.
Formally, a name substitution $\USn ::= \overline{m.n=N}$ can be applied
to a semantic object $x$ with notation $\substw{x}{\USn}$ (the
same notation as module substitution).  The application is also
functorial; here are the important cases:
\[
\begin{array}{rcll}
  \substw{\nhv{m.n}}{\cdot} &=_\UN& \nhv{m.n} \\
  \substw{\nhv{m.n}}{\subst{m.n}{N}, \USn} &=_\UN& N\\
  \substw{\nhv{m.n}}{\subst{m'.n'}{N}, \USn} &=_\UN& \substw{\nhv{m.n}}{\USn} \quad (m.n \neq m'.n')\\
  \substw{(M.n)}{\USn} &=_\UN& M.n\\
  \substw{\overline{N}}{\USn} &=_\UNs& \overline{\substw{N}{\USn}}\\
  \substw{\left\{ \overline{m : \Umty} \right\}}{\USn} &=_\Sigma& \{ \overline{m : \substw{\Umty}{\USn}} \} \\
  \substw{(\UobjTau{\UNs}{\overline{\Uty}})}{\USn} &=_\Umty&
  \UobjTau{\substw{\UNs}{\USn}}{\overline{\substw{\Uty}{\USn}}} \\
\end{array}
\]
\end{definition}

\section{Assumed Haskell judgments}

\begin{figure}
\begin{centering}
\fbox{$\Gamma; \Theta; \theta; \Delta; \impctx; p_0; m \vdash \Uhsbody : \Umty$}\\
\fbox{$\Gamma; \Theta; \theta; \Delta; \impctx; p_0; m \vdash \Uhssig$} \\
\fbox{$ty <: ty'$} \\
\end{centering}
\caption{Assumed Haskell judgments.}
\label{fig:haskell}
\end{figure}

In Figure~\ref{fig:haskell}, we give the black box judgments which we
assume the compiler provides for us.
The module typechecking judgment synthesizes a module type
for the given $\Uhsbody$. Each entity $n$ freshly defined in this module
is given the original name $\Mod{p_0[\Theta]}{m}.n$, which will appear in
the module type $\Umty$. For example, the type of module \modname{Y}, defined in
the component \cidl{p} from~\ref{fig:linked-example}, exports the freshly
named $\MOD{p}{\substHole{A}, \substHole{B}}{Y}.\texttt{L}$.

The signature judgment does not synthesize a type for a signature $\Uhssig$,
in contrast to the module typechecking judgment.
Instead it merely checks that the given $\Uhssig$ is \emph{consistent} with the
requirement types provided by the local context $\Delta$, which
were nondeterministically chosen.  In a type system where requirement
types from dependencies automatically merge together, there is no
reason to expect the syntactic $\Uhssig$ specifies the full requirement
for all the dependencies of the component (the user may have omitted it
for it to be inferred.)

Finally, the subtyping judgment specifies whether or not a
defined entity specification is a subtype of another specification.
Generally, this is simply equality, except that any type is a subtype
of a same-kinded abstract type (\verb|data A| with no constructors.)

\section{Type lookup and export/signature matching}

We now define the mutually recursive judgments for \emph{type lookup}
(Figure~\ref{fig:ty-instantiate}) and \emph{export/signature matching}
(Figure~\ref{fig:ty-match}), arguably the most important rules of our
type system.  Intuitively, computing the type of $p[S]$ simply involves
taking the component type for $p$, wiring it up according to $S$, and
then reading off the provided types under this wiring diagram.
Formally, the wiring diagram represents a name substitution $\USn$ which
refines the name variables in $p$'s type; we must compute this
substitution by export/signature matching.  The export matching rule, in
turn, requires us to compute the types of the module identities in $S$
(we are traversing the whole wiring diagram).
Assuming that module identities are finite, this process is guaranteed
to terminate.

The rules for type lookup are fairly straightforward.  There are special
cases for looking up (1) the component type of the local component
$p_0$, (2) the module type of a module variable $\hv{m}$, and (3) the
Haskell type of a name variable $\nhv{m.n}$.

The rule for export and signature matching is slightly more intricate.
The export matching rule takes as input both the module substitution
being applied $S$, as well as the component requirement $\Sigma_R$ that
the substitution is being applied to.  We first must compute the types
for every module in the substitution (the recursive process), and then
compute an appropriate name substitution $\USn$ induced by the entire
substitution which unifies the export lists (intuitively, we just look
at the export lists and unify the ones which have the same $n$).  Finally, we perform
signature matching, which simply checks that required type $\Umty$,
after name substitution, is consistent with all the types in the context.
It does this by going through each export of the requirement which is
locally defined and checking the local type.

\begin{figure}
\fbox{$\Gamma; \Theta; \theta; \Delta; p_0 \vdash \UP : \Sigma_P$}
\[
\frac{
\begin{array}{c}
p \neq p_0  \qquad  p : \forall \Theta_R.\, \forall \theta_R.\, \Sigma_R \rightarrow \Sigma_P \in \Gamma \\
\JMatch{S}{\Theta_R}{\theta_R}{\Sigma_R}{\USn} \\
\end{array}
}{
\Gamma; \Theta; \theta; \Delta; p_0 \vdash \icid{p}{S} : \substww{\Sigma_P}{S}{Sn}
}
\]
\[
\Gamma; \Theta; \theta; (\Sigma_R; \Sigma_P); p_0 \vdash p_0[\overline{\subst{m_i}{\hv{m_i}}}] : \Sigma_P
\]
\\

\fbox{$\Gamma; \Theta; \theta; \Delta; p_0\vdash M : \Umty$}
\[
\frac{
\Gamma; \Theta; \theta; \Delta; \Up_0 \vdash \UP : \Sigma_P \qquad
m : \Umty \in \Sigma_P
}{
\Gamma; \Theta; \theta; \Delta; \Up_0 \vdash \Mod{\UP}{m} : \Umty
}
\]

\[
\frac{
\hv{m} \in \Theta \qquad
\Delta = \Sigma_R; \Sigma_P \qquad
m : \Umty \in \Sigma_R
}{
\Gamma; \Theta; \theta; \Delta; \Up_0 \vdash \holevar{m} : \Umty
}
\]
\\

\fbox{$\Gamma; \Theta; \theta; \Delta; \Up_0 \vdash \UN :: \Uty$}
\[
\frac{
\Gamma; \Theta; \theta; \Delta; \Up_0 \vdash M : \Umty \qquad
n :: \Uty \in \Umty
}{
\Gamma; \Theta; \theta; \Delta; \Up_0 \vdash M.n :: \Uty
}
\]

\[
\frac{
\begin{array}{c}
\nhv{m.n} \in \theta \qquad
\Gamma; \Theta; \theta; \Delta; \Up_0 \vdash \holevar{m} : \Umty \qquad
n :: \Uty \in \Umty
\end{array}
}{
\Gamma; \Theta; \theta; \Delta; \Up_0 \vdash \nhv{m.n} :: \Uty
}
\]
\caption{Rules for type lookup}
\label{fig:ty-instantiate}
\end{figure}

\begin{figure}

\fbox{$\JMatch{S}{\Theta_R}{\theta_R}{\Sigma_R}{\USn}$}
\[
\frac{
% \begin{array}{c}
% S = \overline{\subst{m_i}{M_i}} \qquad
% \forall i. \quad \Gamma; \Theta; \theta; \Delta; p_0 \vdash M_i : \Umty'_i \\
% \Tsubg{\cdot}{\overline{\tilde{\Umty}'_i <:_{m_i} \tilde{\Umty}_i}}{Sn} \\
% \forall i. \quad \Gamma; \Theta; \theta; \Delta; p_0 \vdash \substww{\Umty_i}{S}{Sn}~\textsf{valid} \\
% \end{array}
  \begin{array}{@{}l@{\;=\;}l@{}}
    S & (\overlinei{\subst{m_i}{M_i}}) \\
    \Theta_R & \{\overlinei{m_i}\} \\
    \theta_R & \Fdom{\USn} \\
  \end{array}
  \quad
  \forall i:\;
    \begin{cases}
      \Gamma; \Theta; \theta; \Delta; p_0 \vdash M_i : \Umty'_i \\
      \Fexports{\Umty_i'} \supseteq \Fexports{\substww{\Umty_i}{S}{\USn}} \\
      \Gamma; \Theta; \theta; \Delta; p_0 \vdash \substww{\Umty_i}{S}{Sn}~\textsf{valid} \\
    \end{cases}
}{
\JMatch{S}{\Theta_R}{\theta_R}{(\overlinei{m_i:\Umty_i})}{\USn}
}
\] \\

% \fbox{$\Tsubg{\USn}{\overline{\Tmatch{\I{Ns}}{m}{\I{Ns}'} }}{\USn'}$}
% \[
% \frac{
% \begin{array}{c}
% \substw{\UNs'}{\USn} = \overline{\nhv{m.n_i}}, \overline{N_j} \\
% \overline{N_j} \subseteq \UNs \qquad
% \forall i. \quad n_i \mapsto N_i \in \UNs \\
% \end{array}
% }{
% \Tsubg{\USn}{\Tmatch{\UNs}{m}{\UNs'}}{\USn, \overline{\nhv{m.n_i} = N_i}}
% }
% \]
% \[
% \frac{
% \begin{array}{c}
% \Tsubg{\USn}{\overline{ \Tmatch{\I{Ns}_i}{m}{\I{Ns}'_i} }}{\USn'} \\
% \Tsubg{\USn'}{\overline{ \Tmatch{\I{Ns}_j}{m}{\I{Ns}'_j} }}{\USn''}
% \end{array}
% }{
% \Tsubg{\USn}{   \Tmatch{\I{Ns}_i}{m}{\I{Ns}'_i} ;  \Tmatch{\I{Ns}_j}{m}{\I{Ns}'_j}  }{\USn''}
% }
% \]\\

\fbox{$\Gamma; \Theta; \theta; \Delta; \Up_0 \vdash \Umty~\textsf{valid}$}
\[
\frac{
\forall i: \quad \Gamma; \Theta; \theta; \Delta; \Up_0 \vdash M_i.n_i :: ty_i' \quad \land \quad
ty_i' <: ty_i
}{
\Gamma; \Theta; \theta; \Delta; \Up_0 \vdash \UobjTau{\overline{M_i.n_i}, \overline{N_j}}{ \overline{n_i :: ty_i} }~\textsf{valid}
}
\]

\caption{Rules for export matching and signature matching.  $\Fexports{\Umty}$ refers to the
the export specification of the type $\Umty$.}
\label{fig:ty-match}
\end{figure}

% A declaration $n$
%   is assigned identity $M.n$ in $\Uhsbody$, but $\{m.n\}$ in $\Uhssig$.
%   The subtyping relation on types is simply equality, except that
%   concrete types are subtypes of abstract types when the kinds are equal.
%   Note that imports in the source code are resolved by first checking if
%   the module is defined in $\Delta$, and otherwise consulting $\impctx$.

%   \begin{figure}
%   \fbox{$\Gamma; \Theta; \theta; \Delta; \Up_0 \vdash P~\textsf{requires}~m : \Umty_R$}
%   \[
%   \frac{
%   \begin{array}{c}
%   \Gamma; \Theta; \theta; \Delta; \Up_0 \vdash p : \forall\Theta'.\,\forall\theta'.\,\Sigma_R \rightarrow \Sigma_P \\
%   \Gamma; \Theta; \theta; \Delta; p_0 \vdash S : \Sigma_R \induces \USn \\
%   m : \Umty_R \in \Sigma_R
%   \end{array}
%   }{
%   \Gamma; \Theta; \theta; \Delta; \Up_0 \vdash p[S]~\textsf{requires}~m : \Umty_R
%   }
%   \]
%   \caption{Signature merging}
%   \end{figure}

\begin{figure}[t]

\fbox{$\shctx; \Gamma; \Theta; \Up_0 ; \theta; \Sigma_R \vdash \{ \Sigma_P; \impctx \} \quad \Uudecl \quad \{ \Sigma_P'; \impctx' \}$}
\[
\frac{
  \begin{array}{c}
    % \Delta = \Sigma_R; \Sigma_P
    % \qquad
    \Gamma; \Theta; \theta; (\Sigma_R; \Sigma_P); \impctx; \Up_0; m \vdash \Uhsbody : \Umty
  \end{array}
}{
  \shctx; \Gamma; \Theta; \Up_0 ; \theta; \Sigma_R \vdash
  \declst%
    {\theta}%
    {\Sigma_P}%
    {\impctx}
  \UsynMod{m}{\Uhsbody}
  \declst%
    {\theta}%
    {\Sigma_P, m : \Umty}%
    {\impctx}
}
\]
\[
\frac{
  % \begin{array}{ll}
  %   % \Delta = \Sigma_R; \Sigma_P
  %   % \qquad
  %   %\hv{m} \in \Theta \\
  %   \Gamma; \Theta; \theta; (\Sigma_R; \Sigma_P); \impctx; \Up_0; m \vdash \Uhssig : \exists \theta_0.\, \Umty_0
  %   &
  %   %\forall i.\quad \Gamma; \Theta; \theta; \Delta; \Up_0 \vdash P_i~\textsf{requires}~m'_i : \Umty'_{Ri} \\
  %   %\Umty_R \oplus \overline{\Umty'_{Ri}} = \exists \theta'.\, \Umty
  %   S_0 = (\subst{m}{\hv{m}})
  %   \\
  %   \JMatch[\Gamma;\Theta;\thetaOR;(\SigmaOR;\cdot);\Up_0]{S_0}{\hv{m}}{\theta_0}{(m:\Umty_0)}{\USn_0}
  %   &
  %   \Umty'_0 = \substww{\Umty_0}{S_0}{\USn_0}
  %   \\
  %   \forall i. \quad \JReqType[\Gamma;\Theta;\thetaOR;(\SigmaOR;\cdot);\Up_0]{m_i}{P_i}{\Umty_i}
  %   &
  %   \Umty' = \bigoplus_i \Umty_i
  %   \\
  %   \multicolumn{2}{c}{\Umty = \Ftrim{m}{\Umty'_0 \oplus \Umty'}}
  % \end{array}
  \begin{array}{c}
    \Gamma; \Theta; \theta; (\Sigma_R;\Sigma_P); \impctx; \Up_0; m
      \vdash \Uhssig
  \end{array}
}{
  \shctx; \Gamma; \Theta; p_0 ; \theta; \Sigma_R \vdash
    \declst%
      {\theta}%
      {\Sigma_P}%
      {\impctx}
    \;\UsynSig{m}{\Uhssig}\;
    \declst%
      {\theta}%
      {\Sigma_P}%
      {\impctx}
}
\]
\[
\frac{
  \begin{array}{c}
    p \haspr \lctxpair{\provs}{\reqs'} \in \shctx
    \qquad
    \Gamma; \Theta; \theta; \Delta; \Up_0 \vdash p[S] : \Sigma
    \\
    r = \overline{m \mapsto m'}
    \qquad
    \overline{M} = \overline{\substw{\provs(m')}{S}}
    % \reqs' \subseteq \textsf{dom}(\Delta)
  \end{array}
}{
  \shctx; \Gamma; \Theta; \Up_0 ; \theta; \Sigma_R \vdash
  \declst%
    {\theta}%
    {\Sigma_P}%
    {\impctx}
  \UsynDep{p[S]}{r}
  \declst%
    {\theta}%
    {\Sigma_P}%
    {\impctx, \overline{m \mapsto M}}
}
\]\\

\fbox{$\shctx; \Gamma; \Theta; \Up_0 ; \theta; \Sigma_R \vdash \{ \Sigma_P; \impctx \} \quad \overline{\Uudecl} \quad \{ \Sigma_P'; \impctx' \}$}
\[
\frac{}{
  \shctx; \Gamma; \Theta; \Up_0 ; \theta; \Sigma_R \vdash \{\Sigma_P;\impctx\} \quad \cdot \quad \{\Sigma_P; \impctx \}
}
\]
\[
\frac{
  \begin{array}{c}
    \shctx; \Gamma; \Theta; \Up_0 ; \theta; \Sigma_R \vdash \{\Sigma_P;\impctx\} \quad mdecl \quad \{\Sigma_P'; \impctx' \} \\
    \shctx; \Gamma; \Theta; \Up_0 ; \theta; \Sigma_R \vdash \{\Sigma_P';\impctx'\} \quad \overline{mdecl'} \quad \{\Sigma_P''; \impctx'' \}
  \end{array}
}{
  \shctx; \Gamma; \Theta; \Up_0 ; \theta; \Sigma_R \vdash \{\Sigma_P;\impctx\} \quad mdecl, \overline{mdecl'} \quad \{\Sigma_P''; \impctx'' \}
}
\]\\

% \fbox{$\JReqType{m}{P}{\Umty}$} \\
% \Scott{Typing of (external) requirement}
% \[
% \frac{
%   \begin{array}{c}
%     p \neq p_0 \qquad
%     p : \forall \Theta_R.\, \forall \theta_R.\, \Sigma_R \rightarrow \Sigma_P \in \Gamma \\
%     \JMatch{S}{\Theta_R}{\theta_R}{\Sigma_R}{\USn} \\
%   \end{array}
% }{
%   \JReqType{m}{p[S]}{\substww{\Sigma_R(m)}{S}{\USn}}
% }
% \]\\

\fbox{$\shctx; \Gamma \vdash \Uunit : \Uunitty$}
\[
\frac{
  \begin{array}{@{}l@{}}
    (\forall\theta.\,\Sigma_R) ~\textsf{most general s.t.}\\
    \;
    \begin{cases}
      \Theta = \Fdom{\Sigma_R}
        \;\wedge\; \theta = \Ffnv{\Sigma_R}
        \;\wedge\; \forall \nhv{m.n} \in \theta: \hv{m} \in \Theta \\
      \shctx; \Gamma; \Theta; p ; \theta; \Sigma_R \vdash
        \{\cdot; \cdot\} \; \overline{\Uudecl} \; \{\Sigma_P; \impctx\}
    \end{cases}
    %\shctx; \Gamma; \Theta; p \vdash \declst{\cdot}{\cdot}{\cdot} \quad \overline{\Uudecl} \quad \declst{\theta}{\Sigma_R; \Sigma_P}{ \impctx}
    
  \end{array}
}{
  \begin{array}{c}
    \shctx; \Gamma \vdash
    \UsynUnit{p}{\Theta}{\overline{\Uudecl}} : \forall\Theta.\,\forall\theta.\,\{\Sigma_R\} \rightarrow \{\Sigma_P\}
  \end{array}
}
\]\\

% \[
% \begin{array}{l}
%   \Ftrim{m}{\UobjTau{\UNs}{\overline{\Uty}}} \quad =
%   \\
%   \qquad
%   \textsf{iface}~(\UNs)~
%     \left\{\begin{array}{@{}c@{\;\;}|@{\;\;}c@{}}
%       \Uty
%       &
%       \begin{array}{@{}l@{}}
%         \Uty\in\overline{\Uty},\\
%         (\Uty ~\textsf{named}~ n) \Rightarrow \UNs(n) = \nhv{m.n}\\
%       \end{array}
%     \end{array}\right\}
% \end{array}
% \]

% \Red{Definition of $\Umty \oplus \Umty$ \dots}

\fbox{$\square \;\le\; \square$ \quad \textsf{less general than}}
\[
  \forall\theta_1.\Sigma_1 \;\;\le\;\; \forall\theta_2.\Sigma_2
  \quad\Leftrightarrow\quad
  \exists \USn :\;
    \begin{cases}
      \Fdom{\USn} = \theta_2 \\
      \Sigma_1 \;\le\; \substw{\Sigma_2}{\USn}
    \end{cases}
\]
\[
\begin{array}{@{}r@{\;\;}c@{\;\;}l@{\;\;}c@{\;\;}l}
  (\overline{m':\Umty'},\overline{m:\Umty_1})
    &\le&
    (\overline{m:\Umty_2})
      &\Leftrightarrow&
      \overline{\Umty_1 \le \Umty_2}
  \\
  \UobjTau{\UNs',\UNs}{\overline{\Uty'},\overline{\Uty_1}}
    &\le&
    \UobjTau{\UNs}{\overline{\Uty_2}}
      &\Leftrightarrow&
      \overline{\Uty_1 <: \Uty_2}
\end{array}
\]

\caption{Rules for \unit{} language.}
\label{fig:unit-ty-rules}
\end{figure}

\section{Typechecking a mixed component}

The rest of the typechecking process (Figure~\ref{fig:unit-ty-rules})
involves processing each declaration of a \unit{} in order, modifying
the local contexts as we process the each declaration of the \unit{}.
Most per-declaration rules are fairly straightforward: module
typechecking produces a $\Umty$ which is added to $\Sigma_P$ and signature
typechecking just checks for consistency with $\Sigma_R$.

However, the dependency rule is a little subtle.  First, we eagerly
typecheck $p[S]$ to ensure that it indeed has a type; not because we
need to use the type, but because there is no guarantee that $p[S]$
is a well-typed instantiation; we need to check.  Second, we need
to consult the shape context in order to determine what set of modules
(substituted according to $S$) we actually bring into scope for import.

Finally, the overall rule for $\mcomp$ nondeterministically synthesizes
the local requirement context $\Sigma_R$ and then runs successively
typechecks each declaration, starting with the empty local contexts.
The results are all bundled up into the final component type of a component.
The synthesized requirement has the following properties: (1) it is specific
enough to permit the component to typecheck, (2) it doesn't contain
unnecessary exports or type declarations, and (3) it is as abstract as possible
with respect to original names (the more name variables, the better.)
