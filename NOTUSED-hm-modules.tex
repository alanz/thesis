\documentclass{article}

\usepackage{mdframed}

%!TEX root = paper.tex
\usepackage[T1]{fontenc} % for tt curly braces
\usepackage{etoolbox} % for various programming stuff
\usepackage{amsmath}
\usepackage{listings}
\usepackage{array}
\usepackage{relsize}
\usepackage{amsthm}
\usepackage{amssymb}
\usepackage{color}
\usepackage[dvipsnames]{xcolor}
\usepackage{graphicx}
\usepackage{booktabs}
\usepackage{fancyvrb}
\usepackage{float}
\usepackage{stringstrings}
\usepackage{stmaryrd}
\usepackage{verbatimbox}
\usepackage{alltt}
\usepackage{multicol}
\usepackage{fixltx2e}
\usepackage[nooneline,bf]{subfigure}
% \usepackage[labelformat=simple]{subcaption}
% \renewcommand\thesubfigure{(\alph{subfigure})}
\usepackage{mdframed}
\usepackage{xcolor}
\usepackage{url}
\usepackage{xspace}
%\renewcommand*{\ttdefault}{txtt}

\lstset{basicstyle=\ttfamily,breaklines=true,columns=fullflexible}

\newmdenv[
    hidealllines=true,
    backgroundcolor=black!20,
    skipbelow=\baselineskip,
    skipabove=\baselineskip
]{greybox}

\newcommand{\ie}{\emph{i.e.},\xspace}
\newcommand{\eg}{\emph{e.g.},~}
\newcommand{\etal}{\emph{et al.}}

\theoremstyle{definition}
\newtheorem{axiom}{Axiom}
\newtheorem{theorem}{Theorem}
\newtheorem{definition}{Definition}
\newtheorem{lemma}{Lemma}

\newcommand{\I}[1]{\ensuremath{\mathit{#1}}}
\newcommand{\cL}{{\calL}}
\newcommand{\Red}[1]{{\color{red} #1}}
\newcommand{\Blue}[1]{{\color{blue} #1}}
\newcommand{\Updated}[1]{\colorbox{red}{Updated! #1}}

% \newcommand{\Backpack}{\texttt{i}Backpack}%
\newcommand{\OldBackpack}{Backpack'14}%
\newcommand{\Backpack}{Backpack'16}%
\newcommand{\Ccomp}{Resolved component}%
\newcommand{\ccomp}{resolved component}%
\newcommand{\Unit}{Mixed component}%
\newcommand{\unit}{mixed component}%
\newcommand{\Iunit}{Instantiated component}%
\newcommand{\iunit}{instantiated component}%
\newcommand{\cid}{component identifier}%
\newcommand{\Cid}{Component identifier}%
\newcommand{\uid}{unit identifier}%
\newcommand{\Uid}{Unit identifier}%
\newcommand{\ir}{linked intermediate representation}
\newcommand{\Ir}{Linked intermediate representation}

% macros for unit syntax and semantic objects
% ------------------------------
% metavars
\newcommand{\Up}{p}
\newcommand{\UP}{P}
\newcommand{\Un}{n}
\newcommand{\UN}{N}
\newcommand{\UNs}{\mathit{Ns}}
\newcommand{\USn}{\mathit{Sn}}
\newcommand{\Ureqs}{\mathit{mreqs}}
\newcommand{\Uunit}{\mathit{mcomp}}
\newcommand{\mprog}{\mathit{mprog}}
\newcommand{\mcomp}{\mathit{mcomp}}
%\newcommand{\Uudecls}{\mathit{ldecls}}
\newcommand{\Uudecl}{\mathit{mdecl}}
\newcommand{\mdecl}{\mathit{mdecl}}
\newcommand{\mdecls}{\mathit{mdecls}}
\newcommand{\marg}{\mathit{marg}}
\newcommand{\Uhsexps}{\mathit{hsexps}}
\newcommand{\Uhsbody}{\mathit{hsbody}}
\newcommand{\Uhssig}{\mathit{hssig}}
\newcommand{\Uunitty}{\Xi}
\newcommand{\Umty}{T}
\newcommand{\Urole}{R}
\newcommand{\Uty}{\mathit{decl}}
\newcommand{\Utys}{\mathit{decls}}
\newcommand{\Udfun}{\texttt{\$} \I{fn}}
\newcommand{\Uinst}{\mathit{inst}}
\newcommand{\Uinsts}{\mathit{insts}}
\newcommand{\Uimp}{\mathit{imp}}
\newcommand{\Uimps}{\mathit{imps}}
\newcommand{\Ufinst}{\mathit{finst}}
\newcommand{\Ufinsts}{\mathit{finsts}}
\newcommand{\Umatch}{\mathit{sub}}
% syntax
\newcommand{\UsynUnitH}[2]{\ensuremath{\texttt{component}~#1~#2}}
\newcommand{\UsynUnit}[3]{\ensuremath{\UsynUnitH{#1}{#2}~\{#3\}}}
\newcommand{\UsynMod}[2]{\ensuremath{\texttt{module}~\modnameif{#1}~\{~#2~\}}}
\newcommand{\UsynSig}[2]{\ensuremath{\texttt{signature}~\modnameif{#1}~\{~#2~\}}}
\newcommand{\UsynSigWith}[3]{\ensuremath{\texttt{signature}~\modnameif{#1}~\{~#2~\}~\textsf{with}~#3}}
\newcommand{\UsynDep}[2]{\ensuremath{\texttt{dependency}~#1~(#2)}}
\newcommand{\UsynDepEmpty}[1]{\ensuremath{\texttt{dependency}~#1}}
% semantic objects
\newcommand{\UobjIface}{\texttt{iface}}
\newcommand{\UobjTau}[2]{\ensuremath{\UobjIface\:(#1)~\{#2\}}}
\newcommand{\UobjUnitTy}[4]{\ensuremath{\forall~#1.\, \forall~#2 .\, #3 \rightarrow #4}}
\newcommand{\UobjUnitTyN}[3]{\ensuremath{\forall~#1 .\, #2 \rightarrow #3}}
% ------------------------------



\newcommand{\doubleplus}{\ensuremath{\mathbin{+\mkern-8mu+}}}
\newcommand{\scomp}[2]{\ensuremath{#2 \circ #1}}

\newcommand{\modname}[1]{\ensuremath{\texttt{#1}}}%
\newcommand{\modnameif}[1]{%
  % make the argument a modname if it is not a macro (invocation)
  % and if it is capitalized
  \ifdefmacro{#1}%
    {#1}%
    {%
      \testcapitalized{#1}
      \ifcapitalized 
        \modname{#1}
      \else
        #1
      \fi%
    }%
  % % make the argument a modname if it is not a (primed) "m" or "n" metavar
  % \ifboolexpr{ test {\ifstrequal{#1}{m}} or %
  %              test {\ifstrequal{#1}{m'}} or %
  %              test {\ifstrequal{#1}{m''}} or %
  %              test {\ifstrequal{#1}{n}} or %
  %              test {\ifstrequal{#1}{n'}} or %
  %              test {\ifstrequal{#1}{n''}} }%
  %   {#1}
  %   {\modname{#1}}
}


\newcommand{\hole}[1]{\ensuremath{#1}}%
% \newcommand{\holevar}[1]{\ensuremath{\colorbox{Apricot}{\ensuremath{#1}}}}%
% \newcommand{\holevar}[1]{\ensuremath{\colorbox{Apricot}{\ensuremath{\modnameif{#1}}}}}
\newcommand{\holevar}[1]{\ensuremath{\langle\modnameif{#1}\rangle}}
\newcommand{\nholevar}[1]{\ensuremath{\{\modnameif{#1}\}}}
\newcommand{\hv}[1]{\holevar{#1}}
\newcommand{\nhv}[1]{\nholevar{#1}}

\newcommand{\cidl}[1]{\ensuremath{\mbox{\texttt{#1}}}}%
\newcommand{\cidlif}[1]{
  % make the argument a component ID if it is not a (primed) "p" or "q" metavar
  % NOTE: sometimes "p" is used as a literal! in such cases, use \cidl instead
  \ifdefmacro{#1}%
    {#1}%
    {%
      \ifboolexpr{ test {\ifstrequal{#1}{p}} or %
                   test {\ifstrequal{#1}{p'}} or %
                   test {\ifstrequal{#1}{p''}} or %
                   test {\ifstrequal{#1}{q}} or %
                   test {\ifstrequal{#1}{q'}} or %
                   test {\ifstrequal{#1}{q''}} }%
        {#1}
        {\cidl{#1}}%
    }
}
\newcommand{\icid}[2]{\cidlif{#1}[#2]}
\newcommand{\uidl}[2]{%
  \ensuremath{
    \cidl{#1}%
    [%
      % if subst is non-empty, allow line break after open paren
      \ifstrempty{#2}{}{\allowbreak}%
      #2
    ]%
  }}

% \newcommand{\Mod}[2]{\ensuremath{#1\!:\!#2}}%
\newcommand{\modcolon}{\hspace{.1ex}{:}\hspace{.15ex}}
\newcommand{\Mod}[2]{\ensuremath{#1\modcolon\modnameif{#2}}}%
\newcommand{\MOD}[3]{\ensuremath{\icid{\cidl{#1}}{#2}\modcolon\modnameif{#3}}}%
\newcommand{\subst}[2]{\ensuremath{\modnameif{#1} \mathop{=}\allowbreak #2}}
\newcommand{\substMod}[3]{\subst{#1}{\Mod{#2}{#3}}}
\newcommand{\substMOD}[4]{\subst{#1}{\MOD{#2}{#3}{#4}}}
\newcommand{\substHole}[1]{\subst{#1}{\holevar{#1}}}
\newcommand{\substw}[2]{#1\llbracket#2\rrbracket}
\newcommand{\substww}[3]{#1\llbracket#2\rrbracket\llbracket#3\rrbracket}
\newcommand{\prov}[2]{\ensuremath{\modnameif{#1} \,\mapsto\,\allowbreak #2}}
\newcommand{\provMod}[3]{\prov{#1}{\Mod{#2}{#3}}}
\newcommand{\provMOD}[4]{\prov{#1}{\MOD{#2}{#3}{#4}}}
\newcommand{\provHole}[1]{\prov{#1}{\holevar{#1}}}
\newcommand{\rename}[2]{\ensuremath{\modnameif{#1} \mapsto \modnameif{#2}}}


% now render unit IDs like
%   \uidl{mylib-3.0-aaa}{
%     \subst{m}{M},                                         % m and M are metavars
%     \substMod{Database}{\uidl{mysql-1.0-ccc}{}}{MySQL}    % Database, mysql..., MySQL are rendered
%    }


\newcommand{\haspr}{\:\triangleright\:}
\newcommand{\shapeis}{\triangleright}
%\newcommand{\lctx}{\mathcal{L}}
\newcommand{\lctx}{{\tilde{\Xi}}}
%\newcommand{\provs}{\mathcal{L}_\textsf{prov}}
%\newcommand{\provs}{{\tilde{\Sigma}}_\textsf{P}}
\newcommand{\provs}{{\tilde{\Sigma}}}
\newcommand{\impctx}{\mathcal{L}}
%\newcommand{\reqs}{{\tilde{\Sigma}}_\textsf{R}}
\newcommand{\reqs}{\Theta}
\newcommand{\pre}[1]{\tilde{#1}}
\newcommand{\preprovs}{\mathbb{P}}
\newcommand{\prereqs}{\mathbb{R}}
\newcommand{\preshape}{\mathbb{L}}
\newcommand{\Reqs}{\mathcal{R}}
\newcommand{\preP}{\preprovs}
\newcommand{\preR}{\prereqs}
\newcommand{\preL}{\preshape}
\newcommand{\shctx}{\tilde{\Gamma}}
%\newcommand{\lctxpair}[2]{\langle #1 \,;\, #2 \rangle}
\newcommand{\lctxpair}[2]{\forall #2.\, #1}
%\newcommand{\lctxpair}[2]{\lctxpairx{#1}{#2}}
\newcommand{\lctxpairx}[2]{\forall #2.\, \{ #1 \}}
%\newcommand{\lctxpairx}[2]{\begin{pmatrix}
%\textsf{requires:}& #2 \\
%\textsf{provides:}& #1 \\
%\end{pmatrix}}
\newcommand{\lctxpairex}[2]{\{#2\} &\rightarrow& \{#1\}}
\newcommand{\lift}[1]{\uparrow\!#1}
\newcommand{\shnull}{}
%\newcommand{\shnull}{\cdot}
%\newcommand{\shape}{\tilde{\Xi}}

% curly braces
% \newcommand{\lcb}{{\tt {\char 123}}}
% \newcommand{\rcb}{{\tt {\char 125}}}
\newcommand{\wrapcb}[1]{\texttt{\textbraceleft} #1 \texttt{\textbraceright}}

% S stands for "source", but these are not the true source declarations.

\newcommand{\Scomp}{\I{rcomp}}
\newcommand{\Sincl}{\I{rincl}}
\newcommand{\Sdecl}{\I{rdecl}}
\newcommand{\Scomponent}[2]{\texttt{component}~ #1 ~ \wrapcb{#2}}
\newcommand{\Sinclude}[2]{\texttt{mixin:}~ #1 ~#2}
\newcommand{\Sincludespec}[3]{\texttt{mixin:}~ #1 ~[\texttt{(}#2\texttt{)}]~[\texttt{requires}~ \texttt{(}#3\texttt{)}]}
\newcommand{\Slparen}{\texttt{(}}
\newcommand{\Srparen}{\texttt{)}}
\newcommand{\Sexposed}[2]{\texttt{exposed-module:}~ \modnameif{#1} ~\{ #2 \}}
\newcommand{\Sother}[2]{\texttt{other-module:}~ \modnameif{#1} ~\{ #2 \}}
\newcommand{\Srequired}[2]{\texttt{signature:}~ \modnameif{#1} ~\{ #2 \}}
\newcommand{\Sreexported}[2]{\texttt{reexported-module:}~ \modnameif{#1} ~\texttt{as}~ \modnameif{#2}}
\newcommand{\Sas}[2]{\modnameif{#1} \;\texttt{as}\; \modnameif{#2}}
\newcommand{\Scom}{\texttt{,}\,}

% macros for helper functions
\newcommand{\Freqs}[2][\Delta]{\ensuremath{\mathsf{reqnames}_{#1}(#2)}}
\newcommand{\Fpreshape}[1]{\ensuremath{\mathsf{preshape}(#1)}}
\newcommand{\FpreshapeI}[2][\Delta]{\ensuremath{\mathsf{preshape}_{#1}(#2)}}
\newcommand{\Fprelink}[1]{\ensuremath{\mathsf{merge}(#1)}}
\newcommand{\FprelinkTwo}[2]{\ensuremath{\mathsf{merge}(#1; #2)}}
\newcommand{\Flink}[1]{\ensuremath{\mathsf{link}(#1)}}
\newcommand{\Fsource}[1]{\ensuremath{\mathsf{source}(#1)}}
\newcommand{\Flet}[1]{\ensuremath{\mathsf{bind}(#1)}}
\newcommand{\Fprerename}[3]{\ensuremath{\mathsf{rnthin}(#1; #2; #3)}}
\newcommand{\Frename}[3]{\ensuremath{\mathsf{rnthin}(#1; #2; #3)}}
\newcommand{\Fprovs}[1]{\ensuremath{\mathsf{provs}(#1)}}
\newcommand{\Fdom}[1]{\ensuremath{\mathsf{dom}(#1)}}
\newcommand{\Fexports}[1]{\ensuremath{\mathsf{exps}(#1)}}
\newcommand{\Ftrim}[2]{\ensuremath{\mathsf{trim}_{#1}(#2)}}
\newcommand{\Ffnv}[1]{\ensuremath{\mathsf{fnv}(#1)}}

% notational convenience
\newcommand{\overlinei}[1]{\overline{#1}^i}
\newcommand{\overlinej}[1]{\overline{#1}^j}
\newcommand{\overlinek}[1]{\overline{#1}^k}
\newcommand{\overlinel}[1]{\overline{#1}^l}



\renewcommand{\topfraction}{.85}
\renewcommand{\bottomfraction}{.7}
\renewcommand{\textfraction}{.1}
\renewcommand{\floatpagefraction}{.9}
\renewcommand{\dbltopfraction}{.9}
\renewcommand{\dblfloatpagefraction}{.9}
\setcounter{topnumber}{9}
\setcounter{bottomnumber}{9}
\setcounter{totalnumber}{20}
\setcounter{dbltopnumber}{9}


\newcommand{\Edward}[1]{\textcolor{Red}{\bf[EZY: #1]}}
\newcommand{\Scott}[1]{\textcolor{Plum}{\bf[SK: #1]}}
\newcommand{\Derek}[1]{\textcolor{Blue}{\bf[DD: #1]}}
\newcommand{\Simon}[1]{\textcolor{RedOrange}{\bf[SPJ: #1]}}
\newcommand{\simon}[1]{\Simon{#1}}

% macros for examples

\newcommand{\uidArraysA}{%
  \uidl{arrays-a}{\holevar{Prelude}}%
}
\newcommand{\uidArraysB}{%
  \uidl{arrays-b}{\holevar{Prelude}}%
}
\newcommand{\uidStructures}{%
  \uidl{structures}{\holevar{Prelude}, \holevar{Array}}%
}
\newcommand{\uidStructuresA}{%
  \uidl{structures}{\holevar{Prelude},
                    \Mod{\uidArraysA}{Array}}%
}
\newcommand{\uidStructuresB}{%
  \uidl{structures}{\holevar{Prelude},
                    \Mod{\uidArraysB}{Array}}%
}

\newcommand{\Tmatch}[3]{#1 <:_{#2} #3}
\newcommand{\Tsubg}[3]{\{#1\} \quad #2 \quad \{ #3 \}}
\newcommand{\Tsub}[5]{\{#1\} \quad \Tmatch{#2}{#3}{#4} \quad \{ #5 \}}


\newcommand{\highlight}[1]{%
  \colorbox{yellow!50}{$\displaystyle#1$}}

\newcommand{\graybox}[1]{%
  \colorbox{gray!30}{$\displaystyle#1$}}

\newcommand{\KP}{\mathsf{P}}
\newcommand{\KM}{\mathsf{M}}

\newcommand{\RP}{\mathbb{P}}
\newcommand{\RM}{\mathbb{M}}
\newcommand{\RN}{\mathbb{N}}

\newcommand{\RT}{T}
\newcommand{\RE}{E}

\newcommand{\row}{\textsf{row}}
\newcommand{\rowty}[1]{\llparenthesis{}\,#1\,\rrparenthesis{}}
\newcommand{\record}[1]{\textsf{record}~#1}


\newcommand{\elab}[1]{\ensuremath{\leadsto{} #1}}


\newcommand{\CK}{\mathcal{K}}
\newcommand{\CT}{\mathcal{T}}

\newcommand{\Ictx}{\Gamma_c}
\newcommand{\IGamma}{\Gamma_c}
\newcommand{\IDelta}{\Delta_c}
\newcommand{\Idash}{\vdash_c}
\newcommand{\Impctx}{\impctx_c}
\newcommand{\iprog}{\I{iprog}}
\newcommand{\invariant}{\prec}

\newcommand{\tDelta}{\tilde{\Delta}}
\newcommand{\Quant}{\Theta}
\newcommand{\induces}{\Rightarrow}
% \newcommand{\declst}[3]{\begin{Bmatrix}#1\\#2\\#3\end{Bmatrix}}
\newcommand{\declst}[3]{\begin{Bmatrix}#2\\#3\end{Bmatrix}}
\newcommand{\Ideclst}[2]{\begin{Bmatrix}#1\\#2\end{Bmatrix}}
%\newcommand{\hsubis}{\rightarrow_S}
%\newcommand{\nhsubis}{\rightarrow_{\USn}}


\newcommand{\JMatch}[6]%
  [\Gamma; \Theta; \theta; \Delta; \Up_0]%
  {\ensuremath{#1 \vdash #2 : \forall #3.\, \forall #4.\, #5 \induces #6}}

\newcommand{\JReqType}[4]%
  % [\Gamma; \Up_0; \Theta; \theta; \Delta]%
  [\Gamma; \Theta; \theta; \Delta; \Up_0]%
  {\ensuremath{#1 \vdash #2 \,@\, #3 : #4}}

\newcommand{\thetaOR}{\hat\theta}
\newcommand{\SigmaOR}{\hat\Sigma}

% Alter some LaTeX defaults for better treatment of figures:
    % See p.105 of "TeX Unbound" for suggested values.
    % See pp. 199-200 of Lamport's "LaTeX" book for details.
    %   General parameters, for ALL pages:
    \renewcommand{\topfraction}{0.9}	% max fraction of floats at top
    \renewcommand{\bottomfraction}{0.8}	% max fraction of floats at bottom
    %   Parameters for TEXT pages (not float pages):
    \setcounter{topnumber}{2}
    \setcounter{bottomnumber}{2}
    \setcounter{totalnumber}{4}     % 2 may work better
    \setcounter{dbltopnumber}{2}    % for 2-column pages
    \renewcommand{\dbltopfraction}{0.9}	% fit big float above 2-col. text
    \renewcommand{\textfraction}{0.07}	% allow minimal text w. figs
    %   Parameters for FLOAT pages (not text pages):
    \renewcommand{\floatpagefraction}{0.7}	% require fuller float pages
	% N.B.: floatpagefraction MUST be less than topfraction !!
    \renewcommand{\dblfloatpagefraction}{0.7}	% require fuller float pages

	% remember to use [htp] or [htpb] for placement
        %
\lstset{language=Haskell,keywords={%  
    package, module, signature, link, unit, where, data, import, instance, hiding, type, dependency%
    }%
}%
\lstdefinelanguage{Cabal}
{
  % list of keywords
  morekeywords={
    name,version,library,exposed-modules,signatures,build-depends,main-is,executable,mixins
  },
  alsoletter=-,
  sensitive=false, % keywords are not case-sensitive
  morecomment=[l]{--}, % l is for line comment
  morestring=[b]" % defines that strings are enclosed in double quotes
}


\newcommand{\DIGatoms}{%
    \begin{array}{rcl}
    m & & \mbox{Module name} \\
    p, q & & \mbox{\Cid} \\
    \end{array}
}

\newcommand{\DIGsource}{%
    \begin{array}{rcl}
    \Uhsbody & & \mbox{Module source} \\
    \Uhssig & & \mbox{Signature source} \\
    \end{array}
}

\newcommand{\DIGresolved}{%
    \begin{array}{rcl}
    \Scomp & ::= & \Scomponent{\Up}{\overline{\Sdecl}} \\[0.3em]
    \Sdecl & ::= & \Sinclude{\Up}{\I{rns}} \\
           & |   & \Sexposed{m}{\Uhsbody} \\
           & |   & \Sother{m}{\Uhsbody} \\
           & |   & \Sreexported{m}{m'} \\
           & |   & \Srequired{m}{\Uhssig} \\
    \I{rns} & ::= & [\Slparen\overline{rn}\Srparen]~[\texttt{requires}~ \Slparen\overline{rn'}\Srparen] \\
    \I{rn} & ::= & \Sas{m}{m'} \\
           & |   & m \\
    \end{array}
}

\newcommand{\DIGuid}{%
    \begin{array}{rcll}
      M   &::=& \Mod{\UP}{m} & \text{Module identifier} \\
          &|&   \holevar{m} & \text{Module hole} \\
      \UP &::=& \icid{\Up}{S} & \text{\Uid} \\
      S   &::=& \overline{\subst{m}{M}} & \text{Module substitution} \\
    \end{array}
}

\newcommand{\DIGshape}{%
    \begin{array}{lcll}
    \lctx & ::= & \lctxpairx{\provs}{\reqs} & \mbox{Component shape} \\
    %r_\textsf{R} & ::= & \overline{m \rightarrow m'} & \mbox{Requirement renaming (total)} \\
    %r_\textsf{P} & ::= & \overline{m \rightarrow m'} & \mbox{Provision renaming (partial)} \\
    \reqs  & ::= & \overline{\hv{m}} & \mbox{Required module variables} \\
    \provs & ::= & \overline{\prov{m}{M}}  & \mbox{Provided modules} \\
    \shctx & ::= & \overline{\Up \shapeis \lctx} & \mbox{Component shape context} \\
    \end{array}
}

\newcommand{\DIGmixed}{%
    \begin{array}{rcl}
      \Uunit &::=& \UsynUnit{\Up}{\reqs}{\overline{\Uudecl}} \\
      \Uudecl &::=& \UsynDep{\UP}{r} \\
              &|&   \UsynMod{m}{\Uhsbody} \\
              &|&   \UsynSig{m}{\Uhssig} \\
              %& &   \qquad \textsf{with}~\overline{P\!:\!m} & \quad\text{(Merges)} \\
              %&|&   \UsynLet{m}{M} & \text{Let binding} \\
      %\mprog &::=& \overline{\Uunit} & \text{Mixed program} \\
      r   &::=& \overline{m \mapsto m'} \\
    \end{array}
}

\newcommand{\DIGinterface}{
\begin{array}{rcll}
  %\Uunitty &::=& \forall \Theta.\, \forall \theta.\, \{\Sigma_R\} \rightarrow \{\Sigma_P\}
  %  & \text{Component type} \\
  %\theta &::=& \overline{\hv{m.n}} & \text{Name variable quantifiers} \\
  %\Sigma &::=& \overline{m : \Umty}
  %  & \text{Provided/required types} \\
  %\Gamma &::=& \overline{\Up : \Uunitty} & \text{Component typing context} \\
  \Gamma &::=& \overline{p : \Xi} & \text{Package environment} \\
  \Delta, \Xi &::=& \overline{m : T^s} & \text{Package type} \\
  &&&\\
  \Umty &::=& \UobjTau{\UNs}{\overline{\Uty}~\overline{\Uinst}~\overline{\Uimp}} & \text{Module type} \\
  s &:=& + ~|~ -  & \text{Polarities} \\
  % \tau_R &::=& \exists \overline{\hv{m.n}}.\, \tau  & \text{Signature type} \\
  \UNs &::=& \overline{\UN} & \text{Export specification} \\
  \Uty &::=& & \text{Defined entity spec} \\
       &  |& \texttt{data}~\graybox{n}~\overline{(a :: \kappa)} & \qquad\text{Abstract data declaration} \\
       &  |& \texttt{class}~\graybox{n}~\overline{(a :: \kappa)} & \qquad\text{Abstract type class} \\
       &  |& \texttt{type family}~\graybox{n}~\overline{(a :: \kappa)} :: \kappa~\texttt{where}~\texttt{..} & \qquad\text{Abstract closed type family} \\
       &  |& \texttt{data}~\graybox{n}~\overline{(a :: \kappa)}~\texttt{where}~\I{dinfo}& \qquad\text{Data declaration} \\
       &  |& \texttt{newtype}~\graybox{n}~\overline{(a :: \kappa)} = \I{ntinfo}& \qquad\text{Newtype declaration} \\
       &  |& \texttt{type}~\graybox{n}~\overline{(a :: \kappa)} :: \kappa = \tau & \qquad\text{Type synonym} \\
       &  |& \texttt{class}~\graybox{n}~\overline{(a :: \kappa)}~\texttt{where}~\I{clinfo} & \qquad\text{Type class} \\
       &  |& \texttt{type family}~\graybox{n}~\overline{(a :: \kappa)} :: \kappa~\texttt{where}~\I{tfinfo} & \qquad\text{Closed type family} \\
       &  |& \texttt{type family}~\graybox{n}~\overline{(a :: \kappa)} :: \kappa & \qquad\text{Open type family} \\
       % Patterns omitted
       % Data families omitted
       &  |& n :: \tau& \qquad\text{Term declaration} \\
  % Notes: Why are the axioms/names kept separately from the instance
  % lists, rather than embedded "implicitly" (e.g., like how coercion
  % for newtype is stored)?  Historically, the DFun was always kept
  % separate from the instance declaration.
  \Uinst &::=& \texttt{instance}~n :: \tau & \text{Class instance} \\
         &  |& \texttt{family instance}~n :: \I{fiinfo} & \text{Family instance} \\
  \Uimp  &::=& M & \text{Transitive module import} \\
  &&&\\
  \Un   && \multicolumn{2}{l}{\text{Haskell source-level entity name}} \\
  \UN &::=& M.\Un & \text{Original name} \\
      &|&   \nhv{m.n} & \text{Name hole} \\
  \USn &::=& \overline{\subst{m.n}{N}} & \text{Name substitution} \\
  &&&\\
  a &&& \text{Haskell type variable} \\
  \tau &::=& \cdots N \cdots & \text{Haskell type} \\
  \kappa &::=& \cdots N \cdots & \text{Haskell kind} \\
  \multicolumn{3}{l}{\I{dinfo}, \I{ntinfo}, \I{clinfo}, \I{tfinfo}, \I{fiinfo}} & \text{Haskell declaration metadata} \\
% R &::=& & \text{Haskell role} \\
%   & | & \mathsf{P} & \qquad\text{Phantom} \\
%   & | & \mathsf{R} & \qquad\text{Representational} \\
%   & | & \mathsf{N} & \qquad\text{Nominal} \\
\end{array}
}


\DeclareMathOperator{\dom}{dom}

\newcommand{\ctx}{\Gamma; \Delta; P_0}

\newenvironment{twocol}
  {
    \setlength{\abovedisplayskip}{0pt}
    \setlength{\belowdisplayskip}{0pt}
    \begin{tabular}{ m{0.47\textwidth} m{0.47\textwidth} }
  }
  { \end{tabular} }

\newcommand\defeq{\mathrel{\overset{\makebox[0pt]{\mbox{\normalfont\tiny def}}}{=}}}


\begin{document}

\special{papersize=8.5in,11in}
\setlength{\pdfpageheight}{\paperheight}
\setlength{\pdfpagewidth}{\paperwidth}

\title{HM-ing Modules}

\maketitle

\noindent
LESSONS LEARNED\@:

If the translation is type directed, if you don't have types, you gonna
have a bad time.

When we lift generative abstraction to top level, need to know all
type variables in scope at the time.  Absent types, may be difficult
to know what is available.

\vspace{1em}
\hrule{}
\vspace{1em}


Conventional wisdom states that type inference and module systems do not
mix.  Evidence for this fact includes (1) the presence of a
\emph{subtyping} relation between module signatures---subtyping can
destroy principle types and the ability to do inference, and (2) the
philosophical principle that you really should write explicit signatures
at the module level---that's the whole point!

In this paper, we claim that \emph{complete} type inference is possible
for a module system which only has applicative functors.  We do so by
elaborating our module language to Hindley-Milner with (generative)
algebraic data types and (structural) Leijen-style records.  This
approach is inspired by F-ing modules, but we don't elaborate into
System $\mathrm{F}_\omega$, which does not admit type inference, but HM,
which does.

There are two key ideas: first, the absence of generative functors means
in an F-ing style formalism, existentials can always be lifted to
the top-level.  Thus, rather than repeatedly pack and unpack existentials,
during our elaboration we directly add declarations to the global context
instead of introducing an existential.  Second, the use of row polymorphism
means that there is no need for a subtyping relation between signatures:
thus, the constraints can be solved entirely with unification.

\section{Hindley-Milner with newtypes and row polymorphism}

The target of our elaboration will be Hindley-Milner extended with
top-level declarations for generative type abstraction (i.e., Haskell
\textsf{newtype}, although in our presentation a newtype declaration
comes with a pair of functions witnessing the coercions) and Leijen
records (as described in ``Extensible records with scoped labels'').

\textsf{newtype}s will be used to bootstrap generative type abstraction
in our module language, while Leijen records will be used to encode
modules (which are handled structurally).  For simplicity, we don't
handle algebraic data types, which would require a bit of finesse to
encode into Leijen records.

\[
\begin{array}{lrcll}
\mbox{Term variables} &&\in& x, f \\
\mbox{Type constructors} &&\in& T \\
\mbox{Type variables} &&\in& a, r \\
\mbox{Labels} &&\in& m, n, l \\
\mbox{Kinds} & \kappa &::=& \star & \mbox{Type} \\
       && | & \kappa \rightarrow \kappa & \mbox{Type constructor} \\
       && | & \row & \mbox{Row} \\
\mbox{Monotypes} & \tau, c &::=& a & \mbox{Type variable} \\
      && | & \tau \rightarrow \tau & \mbox{Function type} \\
      && | & \rowty{} & \mbox{Empty row} \\
      && | & \rowty{l : \tau \,|\, c} & \mbox{Row extension} \\
      && | & \record{c} & \mbox{Record constructor} \\
      && | & T~\overline{\tau} & \mbox{Type constructor} \\
\mbox{Type schemes} & \sigma &::=& \forall \overline{a}.\, \tau \\
\mbox{Expressions} & e &::=& x \\
 && | & \lambda x : \tau.\, e \\
 && | & e~e \\
 && | & \{\} \\
 && | & \{ l = e \,|\, e \} \\
 && | & e.l \\
\mbox{Declarations} & \I{decl} &::=& \multicolumn{2}{l}{\mathsf{newtype}~T~\overline{(a : \kappa)} = n_c~\{ n_d : \tau \}} \\
\mbox{Program binding} & \I{bnd} &::& \I{decl} ~|~ f = e \\
\mbox{Programs} & \I{prog} &::=& \overline{\I{bnd}} \\
\end{array}
\]
%
(In principle all type declarations can be floated to the beginning, but
it is convenient to be able to intersperse them, and can be useful if
you allow top-level shadowing.)

Syntax sugar:

\[
\begin{array}{lcl}
\{ l_1 : \tau_1, l_2 : \tau_2, \ldots \} &::=& \record{\rowty{ l_1 : \tau_1 \,|\, \rowty { l_2 : \tau_2 \,|\, \ldots}\,}}
\end{array}
\]
%
Interesting kinding rules:

\[
\Gamma \vdash \{\} : \row
\]

\[
\frac{
\Gamma \vdash \tau : \star \qquad
\Gamma \vdash r : \row
}{
\Gamma \vdash \rowty{l : \tau \,|\, r} : \row
}
\]

\[
\frac{
\Gamma \vdash c : \row
}{
\Gamma \vdash \record{c} : \star
}
\]
%
Interesting typing rules:

\[
\Gamma \vdash \{\} : \{ \}
\]

\[
\frac{
\Gamma \vdash e : \tau \qquad
\Gamma \vdash e' : \record r
}{
\Gamma \vdash \{ l = e \,|\, e' \} : \{ l : \tau \,|\, r \}
}
\]

\[
\frac{
\Gamma \vdash e : \tau \qquad
\Gamma \vdash e' : \record{r}
}{
\Gamma \vdash \{ l = e \,|\, e' \} : \{ l : \tau \,|\, r \}
}
\]

\[
\frac{
\Gamma \vdash e : \{ l : \tau \,|\, r \}
}{
\Gamma \vdash e.l : \tau
}
\]
%
Newtypes introduce constructors and destructors to environment.

\[
\frac{
\Gamma,
(n_c : \forall \overline{(a : \kappa)}.\, \tau \rightarrow T~\overline{a}),
(n_d : \forall \overline{(a : \kappa)}.\, T~\overline{a} \rightarrow \tau)
\vdash \I{prog}
}{
\Gamma \vdash \mathsf{newtype}~T~\overline{(a : \kappa)} = n_c~\{ n_d : \tau \}; \I{prog}
}
\]
%
Equality on record types (Leijen-style):

\[
\frac{
\tau_1 \equiv \tau_2 \qquad
\tau_2 \equiv \tau_3
}{
\tau_1 \equiv \tau_3
}
\]

\[
\frac{
\tau \equiv \tau' \qquad
r \equiv r'
}{
\{ l : \tau \,|\, r \} \equiv \{ l : \tau' \,|\, r' \}
}
\]

\[
\frac{
l \neq l'
}{
\{ l : \tau, l' : \tau' \,|\, r \} \equiv \{ l' : \tau', l : \tau \,|\, r' \}
}
\]

\section{The module language}

\begin{figure}[H]
\[
\begin{array}{lrcll}
\mbox{Identifiers} &  & \in & f, m, n \\
\mbox{Types} & \I{Ty} &::=& \cdots \\
             &        & | & P & \mbox{Type path} \\
\mbox{Expressions} & E &::=& \cdots \\
                   &   & | & P & \mbox{Expression path} \\
\mbox{Paths} & P &::=& n \\
             &   & | & f\, m \\
             &   & | & m.n \\
\mbox{Modules} & M &::=& \overline{B} \\
\mbox{Bindings} & B &::=& n=E \\
                &   & | & \mathbf{type}~n=\I{Ty} \\
                &   & | & \mathbf{newtype}~n=n_c~\{ n_d : \I{Ty} \} \\
\mbox{Signatures} & S &::=& \overline{D} \\
\mbox{Declarations} & D &::=& n :: \I{Ty} \\
                    &   & | & \mathbf{type}~n \\
                    &   & | & \mathbf{type}~n=\I{Ty} \\
                    &   & | & \mathbf{newtype}~n=n_c~\{ n_d : \I{Ty} \} \\
\mbox{Functors} & F &::=& \mathbf{fun}~(m : S) \Rightarrow F \\
                &   & | & M \\
\mbox{Program} & \I{Fs} &::=& \overline{f = F} \\
\end{array}
\]
\end{figure}

\section{Static semantics}


\paragraph{Semantic objects} We start by defining our semantic signatures:

\begin{figure}[H]
\[
\begin{array}{rcl}
\I{ty} &::=& [ = \tau : \kappa ] \\
       & | & [ \tau ] \\
\I{tys} &::=& \{ \overline{n : \I{ty}} \}\\
\I{sig} &::=& \{ \overline{n : \I{ty}} \,|\, r \}\\
\Sigma &::=& \I{sig} \rightarrow \Sigma ~|~ \I{tys} \\
%   \Xi    &::=& \exists (\rho_P : \{ \overline{m = \KM^i \rightarrow \KM} \} )\,
%                        \overline{(\beta_P : \star^j \times \KM \rightarrow \star)}.\,
%                \forall (\rho_R : \{ \overline{m = \KM}^i \})\,
%                        \overline{(\beta_R : \star)}^j. \\
%          &   & \qquad \Sigma_R \rightarrow \Sigma_P \\
\end{array}
\]
\caption{Semantic signatures}
\end{figure}

\paragraph{The rules}

Here is the type system for this language.  Here are all the judgments
we must specify:

\begin{figure}[H]
\[
\begin{array}{ll}
\mbox{Type elaboration} &\Delta; \Gamma \vdash \I{Ty} : \kappa \leadsto \tau \\
\mbox{Expression elaboration} &\Delta; \Gamma \vdash E : \tau \elab{e} \\
\mbox{Path elaboration} & \Delta; \Gamma \vdash P : \Sigma \\
\mbox{Module elaboration} & \Delta; \Gamma \vdash M : \overline{\I{decl}} \Rightarrow \I{tys} \elab{e} \\
\mbox{Binding elaboration} & \Delta; \Gamma \vdash B : \overline{\I{decl}} \Rightarrow \I{tys} \elab{e} \\
\mbox{Signature elaboration} & \Delta; \Gamma \vdash S \leadsto \I{tys} \\
\mbox{Signature declaration elaboration} & \Delta; \Gamma \vdash D \leadsto \I{tys} \\
\mbox{Functor elaboration} & \Delta; \Gamma \vdash F : \overline{\I{decl}} \Rightarrow \Sigma \elab{e} \\
\mbox{Program elaboration} & \Delta \vdash \I{Fs} \elab{\I{prog}} \\
\end{array}
\]
\caption{Judgments for typing}
\end{figure}

\begin{figure}[H]
\fbox{
$\Delta \vdash \I{Fs} \elab{\I{prog}}$
}

\[
\Delta \vdash \varepsilon \elab{\varepsilon}
\]

\[
\frac{
\begin{array}{c}
\Delta; \varepsilon \vdash F : \overline{\I{decl}} \Rightarrow \Sigma \elab{e} \\
\overline{a} = \I{fv}(\Sigma) \\
\Delta, \overline{\I{decl}}, (f : \forall \overline{a}.\, \Sigma) \vdash \I{Fs} \elab{\I{prog}'}\\
\end{array}
}{
\Delta \vdash f = F; \I{Fs}
\elab{\overline{\I{decl}}; f = e; \I{prog}'}
}
\]

\caption{Program elaboration}
\end{figure}

\begin{figure}[H]
\fbox{
$\Delta; \Gamma \vdash F : \overline{\I{decl}} \Rightarrow \Sigma \elab{e}$
}

\[
\frac{
\begin{array}{c}
\Delta; \Gamma \vdash_M M : \overline{\I{decl}} \Rightarrow \I{tys} \elab{e} \\
\end{array}
}{
\Delta; \Gamma \vdash M : \overline{\I{decl}} \Rightarrow \I{tys} \elab{e}
}
\]

\[
\frac{
\begin{array}{c}
\Delta; \Gamma \vdash S \leadsto \{ \overline{n : \I{ty}} \} \\
\I{sig} = \{ \overline{n : \I{ty}} \,|\, r \} \qquad
r~\mbox{fresh} \\
\Delta; \Gamma, m : \I{sig} \vdash F : \overline{\I{decl}} \Rightarrow \Sigma \elab{e} \\
\end{array}
}{
\Delta; \Gamma \vdash (\mathbf{fun} (m : S) \Rightarrow F) : \overline{\I{decl}} \Rightarrow \I{sig} \rightarrow \Sigma \elab{\lambda (m : \I{sig}).\, e}
}
\]

\caption{Functor elaboration}
\end{figure}

\begin{figure}[H]
\fbox{
$\Delta; \Gamma \vdash S \leadsto \I{tys}$
}

\[
\Delta; \Gamma \vdash \varepsilon \leadsto \{ \}
\]

\[
\frac{
\begin{array}{c}
\Delta; \Gamma \vdash D \leadsto \{ \overline{n : \I{ty}} \} \\
\Delta; \Gamma, \overline{n : \I{ty}} \vdash S \leadsto \{ \overline{n' : \I{ty}'} \} \\
\end{array}
}{
\Delta; \Gamma \vdash (D; S) \leadsto \{ \overline{n' : \I{ty}'}, \overline{n : \I{ty}} \}
}
\]
\caption{Signature elaboration}

THIS IS NOT CORRECT, $n$ needs to be brought into scope.
\end{figure}

\begin{figure}[H]
\fbox{
$\Delta; \Gamma \vdash D \leadsto \I{tys}$
}

\[
\frac{
\Delta; \Gamma \vdash \I{Ty} : \star \leadsto \tau
}{
\Delta; \Gamma \vdash n :: \I{Ty} \leadsto \{ n : [ \tau ] \}
}
\]

\[
\frac{
a~\mbox{fresh}
}{
\Delta; \Gamma \vdash \mathbf{type}~n \leadsto \{ n : [ = a : \star ] \}
}
\]

\[
\frac{
\begin{array}{c}
\Delta; \Gamma \vdash \I{Ty} : \star \leadsto \tau \\
\end{array}
}{
\Delta; \Gamma \vdash \mathbf{type}~n = \I{Ty} \leadsto
    \{ n : [ = \tau : \star ]
    \}
}
\]

\[
\frac{
\begin{array}{c}
a~\mbox{fresh} \\
\Delta; \Gamma \vdash \I{Ty} : \star \leadsto \tau \\
\end{array}
}{
\Delta; \Gamma \vdash \mathbf{newtype}~n = n_c~\{ n_d : \I{Ty} \} \leadsto
    \{ n : [ = a : \star ]
     , n_c : [\tau \rightarrow a]
     , n_d : [a \rightarrow \tau]
    \}
}
\]
\caption{Signature declaration elaboration}
\end{figure}

\begin{figure}[H]
\fbox{
$\Delta; \Gamma \vdash M : \overline{\I{decl}} \Rightarrow \I{tys} \elab{e}$
}

\[
\Delta; \Gamma \vdash \varepsilon : \{\} \elab{\{\}}
\]

\[
\frac{
\begin{array}{c}
\Delta; \Gamma \vdash B : \overline{\I{decl}} \Rightarrow \{ \overline{ n : \I{ty} } \} \elab{e_1} \\
\Delta; \Gamma, \overline{\I{decl}}, \overline{n : \I{ty}} \vdash M : \overline{\I{decl'}} \Rightarrow \{ \overline{ n' : \I{ty}'} \} \elab{e_2}
\end{array}
}{
\Delta; \Gamma \vdash (B; M) : (\overline{\I{decl}}, \overline{\I{decl}'}) \Rightarrow \{ \overline{n : \I{ty}}, \overline{n' : \I{ty}'} \}
\elab{ \{ \overline{n' = e_2.n'}, \overline{n = e_1.n} \} }
}
\]
\caption{Module elaboration}
\end{figure}

\begin{figure}[H]
\fbox{
$\Delta; \Gamma \vdash B : \overline{\I{decl}} \Rightarrow \I{tys} \elab{e}$
}

\[
\frac{
\Delta; \Gamma \vdash E : \tau \elab{e}
}{
\Delta; \Gamma \vdash n = E : \{ n : \tau \} \elab{\{ n = [e] \}}
}
\]

\[
\frac{
\Delta; \Gamma \vdash T \leadsto \tau
}{
\Delta; \Gamma \vdash \mathbf{type}~n = \I{Ty} : \{ n : [= \tau : \star] \} \elab{\{ n = [= \tau : \star] \}}
}
\]

\[
\frac{
\begin{array}{c}
T~\mbox{fresh} \\
\Delta; \Gamma \vdash \I{ty} : \star \leadsto \tau \\
\overline{a} = \I{fv}(\Gamma) \\
\end{array}
}{
\begin{array}{l}
\Delta; \Gamma \vdash \mathbf{newtype}~n = n_c~\{ n_d : \I{Ty} \} \\
\quad    : (\mathsf{newtype}~T~\overline{a} = n_c~\{ n_d : \tau \}) \Rightarrow \\
\quad    \{ n : [ = T~\overline{a} : \star ], n_c : \tau \rightarrow T~\overline{a} , n_d : T~\overline{a} \rightarrow \tau \} \\
\elab{\{
n = [ = T~\overline{a} : \star ],
n_c = [ n_c ],
n_d = [ n_d ]
\} }
\end{array}
}
\]

\caption{Binding elaboration}
\end{figure}

\begin{figure}[H]
\fbox{
$\Delta; \Gamma \vdash P : \Sigma \leadsto e$
}

\[
\frac{
\Gamma(n) = \tau
}{
\Delta; \Gamma \vdash n : \tau \elab{n}
}
\]

\caption{Path elaboration}
\end{figure}

In principle, it should be possible to support sealing, but we need a stronger
form of top-level type abstraction (able to abstract over values as well as
types.)

\end{document}
